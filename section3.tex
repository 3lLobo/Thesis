In this section we will go over related work and relevant background information. The depth of the explanation is adopted to the expected prior knowledge of the reader. The reader is supposed to know the basics of machine learning, including probability theory and basic knowledge on neural networks and their different architectures.

\subsection{The Graph VAE – one shot method}

\textbf{VAE}\\
The VAE as first presented by \cite{kingma_auto-encoding_2014} is an unsupervised generative model consisting of an encoder and a decoder. The architecture of the VAE differs from a common autoencoder by having a stochastic module between encoder and decoder. Instead of directly using the output of the encoder, a distribution of the latent space is predicted from which we sample the input to the decoder. The reparameterization trick allows the model to be differentiable. By places the sampling module outside the model we get a deterministic model which can be backpropagated.

\textbf{MLP}\\
The Multi-Layer Perceptron (MLP) was one on the first machine learning models. In its basic structure it takes a one dimensional input, fully-connected hidden layer, activation function and finally output layer with normalized predictions.
Images or higher dimensional tensors can be processed by flattening them to a one dimensional tensor and adjusting the input dimensions. This makes the MLP a flexible and easy to implement model.(reference)
\\
\textbf{Graph convolutions}\\
CNNs have shown great results in the field of images classification and object detection. This is due to the fact that a convolution layer takes into account the relation of one pixel to its neighbors. The same holds for graph CNNs where at each convolution the information of each node is passed on as messages to all its neighbors. Each convolution applies an activation function in between the steps. In this case the information is the edges each node has. To process node attributes and edge atributes we have to look at more complex models \cite{tiao_variational_nodate}. 
\\
\textbf{RGCN}\\
Realtional Graph Convolution Net (RGCN) was presented in \cite{kipf_semi-supervised_2017} for edge prediction. This model takes into account features of nodes. ***Elaborate***
\\
\textbf{Graph VAE}\\
% Encoder options: MLP RGCN
% Decoder MLP
% One shot: creating adjacency and feature matrix at once.
Now we have all the building bocks for a Graph VAE. The encode can either be a MLP, a GCNN or an RGCN. The same holds for the decoder with the addition that model architechture needs to be inverted. An verison of a Graph VAE presented in \cite{simonovsky_graphvae_2018}. This model combines both the previous methods. The input graph undergoes relational graph convolutions before it is flattened and projected into latent space. After applying the reparametrization trick, a simple MLP decoder is used to regenerate the graph. In addition the model concatenates the input with a target vector $y$, which represents ???. The same vector is concatenated with the latent tensor. ***Elborate why they do that***.

Graphs can be generated recursively or in an one-shot approach. This paper uses the second approach and generates the full graph in one go. ***Cite?***
% This model will be the starting point for our research.




\subsection{Permutation Invariance}

% Permutation Invariance
% The position or rotation of a graph can vary. 
% Use graph matching to detect similarities between graphs

Permutation invariance refers to the invariance of a permutation of an object. An visual example is the the image generation of numbers. If the loss function of the model would not be permutation invariant, the generated image could show a perfect replica of the input number but due to positional permutation the loss function would penalize the model. 
OR: An example is in object detection in images. An object can have geometrical permutations such as translation, scale or rotation, none the less the model should be able to detect and classify it. In that case, the model is not limited by permutations and  is there fore permutation invariant.
In our case the object is a graph and the nodes can take different positions in the adjacency matrix. To detect similarities between graphs we apply graph matching.

\textbf{Graph matching algorithms}

% present different ones or only max-pooling?
There are various graph matching algorithms. The one we will implement is the max-pooling (Finding Matches in a Haystack: A Max-Pooling Strategy for Graph Matching in the Presence of Outliers). 

% Max-pooling algorithm comes here !!!
The max-pooling graph matching algorithm presented by

The algorithm returns a symmetric affinity matrix for all nodes

The resulting similarity matrix gives us $X*$ which is continuous and therefore useless. To transform is to a discrete $X$ we use the hungarian algorithm (GPU-accelerated Hungarian algorithms for the Linear
Assignment Problem)

Hungarian algorithm for discrimination of X
\\
second option:
Compare only graph structure. 
NX algorithms: Greedy, Shortest path \dots



\subsection{Knowledge Graphs}

Knowledge Graphs are great! The best in the world.
% Knowledge Graphs which we will be using
% We will focus on the generation of KGs.
% Representation of KG as adjacency, edge feature and node feature matrix
