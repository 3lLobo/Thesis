% !BIB program = bibtex
\documentclass{article}

\usepackage{graphicx, color}

\usepackage[a4paper,margin=2cm]{geometry}

% Here are my package imports:
\usepackage{caption}
\usepackage{subcaption}
\usepackage{hyperref}
\usepackage{amsmath}
\usepackage{amssymb}
\usepackage{algorithm}
\usepackage{algorithmicx}
\usepackage{algpseudocode}
\usepackage{layouts}
\usepackage[table,xcdraw]{xcolor}

\setlength{\parindent}{3em}
\setlength{\parskip}{1em}
\newcommand{\red}[1]{{\color{red}{#1}}}

\RequirePackage[backend=bibtex,style=nature]{biblatex}

\addbibresource{fullrefs.bib}

\begin{document}




\begin{titlepage}



\newcommand{\HRule}{\rule{\linewidth}{0.5mm}} % Defines a new command for the horizontal lines, change thickness here

\center % Center everything on the page

 

%----------------------------------------------------------------------------------------

%	HEADING SECTIONS

%----------------------------------------------------------------------------------------



\includegraphics[width=\linewidth]{data/images/uvaENG}\\[2.5cm]

\textsc{\Large MSc Artificial Intelligence}\\[0.2cm]

\textsc{\Large Master Thesis}\\[0.5cm] 



%----------------------------------------------------------------------------------------

%	TITLE SECTION

%----------------------------------------------------------------------------------------



\HRule \\[0.4cm]

{ \huge \bfseries Knowledge Generation \\[0.4cm] } % Title of your document

\HRule \\[0.5cm]

 

%----------------------------------------------------------------------------------------

%	AUTHOR SECTION

%----------------------------------------------------------------------------------------



by\\[0.2cm]

\textsc{\Large Florian Wolf}\\[0.2cm] %you name

{12393339}\\[1cm]





%----------------------------------------------------------------------------------------

%	DATE SECTION

%----------------------------------------------------------------------------------------



{\Large \today}\\[1cm] % Date, change the \today to a set date if you want to be precise



{48 Credits}\\ %
{April 2020 - December 2020}\\[1cm]
%{Period in which the research was carried out}\\[1cm]%



%----------------------------------------------------------------------------------------

%	COMMITTEE SECTION

%----------------------------------------------------------------------------------------

\begin{minipage}[t]{0.4\textwidth}

\begin{flushleft} \large

\emph{Supervisor:} \\

Dr Peter \textsc{Bloem} \\ Thiviyan \textsc{Singam} \\ Chiara \textsc{Spruijt} % Supervisor's Name

\end{flushleft}

\end{minipage}

~

\begin{minipage}[t]{0.4\textwidth}

\begin{flushright} \large

\emph{Asessor:} \\

Dr Paul \textsc{Groth}\\

\end{flushright}

\end{minipage}\\[2cm]



%----------------------------------------------------------------------------------------

%	LOGO SECTION

%----------------------------------------------------------------------------------------



% \framebox{\rule{0pt}{2.5cm}\rule{2.5cm}{0pt}}\\[0.5cm]

\includegraphics[width=2.5cm]{data/images/uva.png}\\ % Include a department/university logo - this will require the graphicx package

% \textsc{\large \red{institute name}}\\[1.0cm] % 

 

%----------------------------------------------------------------------------------------



\vfill % Fill the rest of the page with whitespace



\end{titlepage}

\tableofcontents
\listoftables
\newpage

\section*{Abstract}
% We generate Knowledge! \cite{kipf_contrastive_2020}

This thesis investigates the idea of having a VAE learn latent features of the raw data from KGs. Building on successful approaches of prior work, an embedding based, a fully-connected and a convolutional model are evaluated on the two most popular KGs. The influence of a permutation invariant loss function on the models capability to generalize on sparse subgraphs is analyzed.
We compare the performance of our models on the task of link prediction to sate of the art scores.
The results compare ???
The model $x$ outperforms the others, indicating that convolutions are [/not] necessary.
Interpolation of the latent space shows that the model learns to featurize latent dimensions???
When generating subgraphs with up to $n$ nodes, we see ???
Finally we filter generated triples for type constrained predicates on subject or object. A $x\%$ of the generated triples adhere to this axiom. Thus we can say that VAE's are to a certain extend able to capture the underlying semantics of a KG.   


% \section*{Notes}
% Here stand written my initial thoughts and notes on the upcoming thesis project.

\subsection*{Semantic Parser}
Triple extraction from text documents.

\begin{itemize}
    \item Stanford Parser
    Parse texat in batches not single with for loop.\\
    Does the parsing happen locally or on their server?\\
    \url{https://github.com/stanfordnlp/stanfordnlp}
    \\
    Load other or your own models:\\
    \url{https://stanfordnlp.github.io/stanfordnlp/models.html}
    \\
    OpenIE uses stanford and adds what value?
    
    \item Allen NLP
    
\end{itemize}

\subsection*{Differential Logic}
Neural Nets made for Logic and differential. This was we can train a deep model on logic and apply it to our extracted triples.
Ideas on what logic we should predict:
\begin{itemize}
    \item \textbf{Relevance} use it as metric how relevant the information might be.
    \item \textbf{Knowledge Base} train it on an actual knowledge base like ConceptNet and evaluate our triples.
\end{itemize}

\hline


\section{Introduction}

% Here comes a beautiful introduction. %Promise!

To begin with, we should clarify the intended ambiguity of this work's title. One could argue, that we live in times of the fastest advances in science and technology in the history of humanity, therefore making us the \textit{Knowledge Generation}. While we continuously keep researching and accumulating knowledge, we have historically not been willing to share our knowledge with any other species in this Universe. All scientific milestones are from us, for us. 

The rise of Artificial Intelligence (AI) marked a turning point of this tradition. For the first time we invest in sharing our knowledge and with systems which can act on superhuman dimension. While machine learning models might not be considered a specie, they have in fact surpassed human intelligence in certain fields (yes, DeepMind's AlphaGO) \cite{silver_mastering_2017}. This is not only seen as progress but also as danger. The world's richest man, Elon Musk, both build his fortune on AI and respects it as humanity's biggest risk. 

\begin{figure}[H]
    \centering
    \includegraphics[height=.21\textwidth, keepaspectratio]{data/images/ElonMusk.png}
    \caption{Elon Musk in a tweet on AI. Source \cite{noauthor_elon_nodate}}
    \label{fig1:Elon}
\end{figure}


% \begin{figure}[H]
%     \centering
%     \includepdf[pages=-,pagecommand={},width=\textwidth]{data/images/Elon75a4.pdf}
%     \caption{Elon Musk Tweet on AI. Source \cite{twitter}}
%     \label{fig1:Elon}
% \end{figure}


Closing the circle to the ambiguity of the title and relating to the popular concern, that AI might reach a point where it does not need humans anymore to keep evolving, we ask the crucial question: Can AI generate knowledge? 


\subsection{Motivation}

Turning away from philosophy, we introduce this thesis in scientific context. A key area of AI is representation learning, where the model learns to identify and disentangle causalities and underlying features of the data. Understanding the semantics of the data is specifically useful for unsupervised learning. 


% computer vision
The task of generating data has been widely explored for images. Computer vision has reached a point where a simple image can be semantically segmented, where objects can be detected and classified and even relations between entities inferred \cite{kipf_contrastive_2020}.


% molecular graphs
Advances in the parallel field of  graph generation have received less attention, yet showed promising results. Data stored as graph has a high density of information and rich semantics, which makes it attractive for variational inference. The recent success of Simonovsky et al. \cite{simonovsky_graphvae_2018} on the generation and completion of molecules represented in graph structure, initially inspired our research. Next to molecules, graphs can be used to store knowledge. While real world Knowledge Graphs (KG) have a far higher complexity than molecule graphs, the proposed generative model, Variational Auto-Encoder (VAE) also has proven its capacity to learn from huge datasets with high variance. Inspired by Simonovsky work and motivated by the vision of generating knowledge, we explore the possibilities and limitation of KG generation with VAEs.



\subsection{Expected Contribution}

The main contributions, which we anticipate for this thesis are firstly the proof of concept that a graph VAE can capture, disentangle and reproduce the underlying semantics of a real world KG. While Simonovsky used small subgraphs with multiple edges, we proof our hypothesis on  generating the smallest possible graph of two nodes, also representable as single triple. The VAE is tested and evaluated in several experiments, including link prediction, latent space interpolation and accuracy of generating valid triples.
% Proof of concept KGVAE


% draw parallels to molecules and explore the differences
We compare our results to related methods in the filed of KGs and investigate the impact of different hyperparameter. The main focus is be on the influence of the graph matching loss function, the encoding through graph convolutions and stochastic inference. Further, we aim to mimic the success of to molecule generation and will therefore continuously point out similarities and differences to our work.

% Implementation of graph matching in batches
On a lower level, we hope to contribute with our implementation of the max-pooling graph matching algorithm for tensor batches. While the algorithm has been cited and implemented numerous times, a working implementation, compatible with deep learning tasks, has to the best of our knowledge not yet been published.  

% Method for syntax cohearence of generated triples from real world KG
Lastly we introduce a high-level method for evaluating the validity of generated data, which compares the type constrain of the generated triple's predicate with its entity types and reports accuracy. This is made possible by expanding the existing dataset FB15K-237 with entity types from its original KG Freebase. While this scoring method is error prone, it does give an insight into the level representational potential of the model and the syntax coherence of the generated triples. Future work can use this evaluation method to track progress and compare to the baseline. 


% This thesis is aimed to be a proof of concept, providing insight into the capability of the VAE on generating KG triples and to indicate if further research in this direction would be meaningful.

\subsection{Research Question}

\begin{center}
    \texttt{How successful is a VAE in representation learning of real world KG compared to molecule graph data and what is the impact of each major hyperparameter?}
    \label{sec1:requestion}
\end{center}


Without further ado -


\section{Related Work}
In this section I will present my literature research up-to-date. The included topics are either relevant as background knowledge or state-of-the-art models.



\subsection{Knowledge}
% Get knowledge out of text.
% Humans read text and understand the knowledge.
% Make text machine readable. new format.
% Get accurate knowledge.
% Get background knowledge.

If we thjnk about, how we aquire knowledge, the most common way is audio-visually, e.g in form of a lecture, a movie, or book. A common way to pass on knowledge is by bringing it to text. For machines to be able to reason and work with knowledge, it needs to be transferred to a machine readable format. The most popular format is a tabular database. A newer database approach is the knowledge graph (KR) which is based on relations between entities. This lets the machine reason on this knowledge, answer complex questions and make conclusions more similar to human thinking.
The format of KG is a triple consisting of a subject, a directed relation and an object.
The task of converting knowledge from text to KG format is non-trivial. In this thesis, we would like to focus on the extraction of accurate knowledge. The idea behind it is that text possesses various levels of knowledge and we, as reader, are biased by our prior knowledge and the intention or drive for reading the text. 

\begin{itemize}
    \item Semantic Parsing\\
    The most old-school and technical approach is to tackle each sentence with NLP methods. This means tagging each word, linking references and finding relations. This will extract all possible triples. This can then be aligned with an existing database \cite{kertkeidkachorn_t2kg_2018}.
    \item Knowledge Graph\\
    Form of representing knowledge or simply a database. Makes information machine readable. It is build of relation triples.
    \item Triple SRO\\
    Triples are formed by two nodes and on link. The starting node is the subject, the second node is the object and the directed link is the relation of the subject with the object. 
    \item Semantic Web\\
    The future of the internet where the data presented on websites can be read and understood by the browser. For this to be possible, websites should present its information in KG format in the metadata.
    \item Existing Methods in NLP\\
    Different semantic parser have been developed. They extract all possible kind of triples from text. While being grammatically correct, the extracted triples do not represent the information a human reader would get by reading the text. The Stanford Parser might be the most popular and advanced one \cite{che_towards_2018}. Further the Allen NLP and OpenIE projects offer powerful parsing tools.
\end{itemize}

\subsection{Embeddings}
% Plain text can not be used as input to a model. We need a valid representation of the words. Word embeddings are generated on the full corpus covering all words which might appear. Each word is represented by a high dimensional vector. this can be used to encode the text as input or decode it back from the model output.
For any model we need an encoding to input text. Word embeddings have established themselves over the last decade as the solution. There are differences in how they are trained , how the relations between words is captured and how the context is represented.
\begin{itemize}
    \item Word Embeddings\\
    word2vec is the most established word embedding, easy to train and implement \cite{mikolov_efficient_2013}. Yet this method is becoming outdated and being replaced by newer solutions. \\
    \\BERT, also by Google, seems to be the sate of the art. It is able to predict words or full sentences as vectors. Using a bidirectional architecture and an attention score for each token, this model is able to catch much more context than its predecessors \cite{devlin_bert_2019}. The attention score indicates how much other words point towards the selected word within a sentence \cite{vaswani_attention_2017}. This way the importance of the words can be compared. The model on it's own is a text classifier but can be tweaked to output word embeddings.
    \item Graph Embeddings\\
    TransE represents entities in in low-dimensional embedding. The relationships between entities are represented by the vector between two entities \cite{bordes_translating_2013}.
    (How are different relation between the same entities represented?)
    \item OntoUSP\\
    This method learns a hierarchical structure to better represent the relations between entities in embedding space.
\end{itemize}

\subsection{Learning Methods}
A big challenge of this field is the lack of labeled data. Public knowledge graphs like DBpedia hold huge collections of knowledge. In my opinion, the size is the problem. The plain amount of data makes it hard to incorporate it in an efficient pipeline. Further, since it needs to cover all topics, the representation becomes less specific.
Looking at the usecases of KGs, most tasks actually do not require this total coverage and instead are topic specific. Thus, we would like to focus on a targeted approach.
\begin{itemize}
    \item Supervised Learning\\
    Requires a labeled dataset. For the case of text to graph the options are limited. Adecorpus provides such a dataset for drug interactions. Facebook research offers a Q&A text to graph dataset called babi. This dataset is a benchmark for question answering algorithms.\\
    \item Distant supervision\\ 
    makes use of the a database like DBpedia and infers labels by comparing the similarity of entities.\\
    \item Contrastive Learning\\
    Can be supervised or unsupervised approach and focuses on similarities between predictions. The loss is computed by the energy function of the output. To me this seems like an interesting approach to create a world model \cite{kipf_contrastive_2020} and it has not yet been applied to plain text.
\end{itemize}


\subsection{Recurrent Graph Models}
The idea of creating a graph recurrently node by node seems intuitive. One recent example of generating a graph using a recurrent VAE architecture was \cite{belli_image-conditioned_2019}. Here the model generates gets bird-view images of roadmaps and generates a graph representation. each generated node is fed back into the encoder as prior and a stop signal is generated once finished.
Applying this to knowledge from text is one of my two main ideas \ref{idea:VAE}.

Another recent approach uses recurrent Graph GAN \cite{li_learning_2018}. Learns distribution of the training graphs. Creates Graph sequentially.

\subsection{Graph Normalizing Flows}
Create Graph all at once. Would it be possible to generate it recursively? Conditioned on query?
\begin{itemize}
    \item unsupervised learning with CFG
    In the application of context free grammar, NFs have been trained on plain text \cite{jin_unsupervised_2019}. The approach was unsupervised with a loss function which takes in account the distribution of the output prediction and the KL divergence between the distributions of each output. This made the model generate outputs with high certainty and which lay far apart from each other. The loss also encourages the output to be close to the input. For the input and output two different embeddings are needed. Thus, we need a similarity measure between them.
    
\end{itemize}

\subsection{Variational Auto Encoder (VAE)}

These models consist of an encode and a decoder. The encoder encodes the input to an low-dimensional latent space. The decoder takes a signal from latent space and reconstructs the input. 
Exploring the latent space makes it oossible to use the decoder as generative model.
The posterior can only be approximated by the ELBO

\section{Background}
In this section we will go over related work and relevant background information for our model and experiments. The depth of the explanation is adopted to the expected prior knowledge of the reader. The reader is supposed to know the basics of machine learning and deep learning, including probability theory and basic knowledge on neural networks and their different architectures. Basic principles such as forward pass, backpropagation and convolutions are expected to be understood. Further the use and functionality of deep learning modules such as the model, the optimizer and the terms target and prediction should be known. This also includes being familiar with the training and testing pipeline of a model in deep learning.

Should all these boxes be checked, then we can expect to get a deeper understanding of the magic behind the VAE and its differences to a normal autoencoder. After that we will present how convolutional layers can be used on graphs. Of course where there is a layer there is a model, thus we are presenting the graph convolutional network (GCN). Closing the circle we show how we can adopt the VAE to graph convolutions. Wrapping things up, we present the state of the art algorithms for graph matching, which will be util to allow permutation invariance when matching prediction and target.  

\subsection{The Graph VAE – one shot method}

\subsubsection{VAE}
% The VAE as first presented by \cite{kingma_auto-encoding_2014} is an unsupervised generative model consisting of an encoder and a decoder. The architecture of the VAE differs from a common autoencoder by having a stochastic module between encoder and decoder. Instead of directly using the output of the encoder, a distribution of the latent space is predicted from which we sample the input to the decoder. The reparameterization trick allows the model to be differentiable. By placing the sampling module outside the model we get a deterministic model which can be backpropagated.



The VAE as first presented by \cite{kingma_auto-encoding_2014} is an unsupervised generative model in form of an autoencoder, consisting of an encoder and a decoder. We will The architecture of the VAE differs from a common autoencoder by having a stochastic module between encoder and decoder. The encoder can be represented as recognition model with the probability $p_{\boldsymbol{\Theta}}(\mathbf{z} \mid x)$ with $x$ being the varbiable we want to inference and $z$ being the latent representation given an observed value of $x$. The encoder parameters are represented by $\Theta$. Sinilarly, We denote the decoder as $p_{\boldsymbol{\Theta}}(\mathbf{x} \mid z)$, which given a latent representation $z$ produces a probability distribution for the possible values, corresponding to the input of $x$. This will be the base architecture of all our models in this thesis.

The main contribution of the VAE is the so called reparameterization trick. By sampling from the latent prior distribution, we get stochastic module inside our model, which can not be backpropagates through and makes machine learning not possible. By placing the stochastic module outside the model FIGURE!!!, we can again backpropagate. We use the predicted latent space as mean and variance for a Gaussian normal distribution, from which we then sample $\epsilon$, which acts as external parameter and does not need to be updated.

This makes the true posterior $p_{\boldsymbol{\theta}}(\mathbf{z} \mid \mathbf{x})$ intractable. Thus, we assume that the prior to the decoder to be Gaussian with an approximately diagonal covariance, which gives us the approximated posterior.

\begin{equation}
    \log q_{\phi}\left(\mathbf{z} \mid \mathbf{x}^{(i)}\right)=\log \mathcal{N}\left(\mathbf{z} ; \boldsymbol{\mu}^{(i)}, \boldsymbol{\sigma}^{2(i)} \mathbf{I}\right)
\end{equation}
    
This gives us computational freedom. meaning we can compute the posterior probability. Using the Monte Carlo estimation of $q_{\phi}(\mathbf{z} \mid \mathbf{x})$ we get the so called estimated lower bound (ELBO):
\begin{equation}
    y_{k}(\mathbf{x}, \mathbf{w})=\sigma\left(\sum_{j=0}^{M} w_{k j}^{(2)} h\left(\sum_{i=0}^{D} w_{j i}^{(1)} x_{i}\right)\right)
\end{equation}
We denote the first term the regularization term, as it forces the model into using a Gaussian normal prior. The second term represents the reconstruction loss, matching the prediction with the target.

While we use a discrete input space, the output space is a continuous probability. To generate final result, the prediction is used as binomial probability distribution, from which we then sample. Once training  of a VAE is completed, the Decoder can be used on its own to generate new samples by using latent input signals \cite{kingma_auto-encoding_2014}.


\subsubsection{MLP}
% History introduction
The Multi-Layer Perceptron (MLP) was the beginning of machine learning models.
% Invented by who when
Its properties as universal approximator has been discovered and widely studied since 1989. The innovation it brought to existing models was the hidden layer between the input and the output.

% Functionality
% In its basic structure it takes a one dimensional input, fully-connected hidden layer, activation function and finally output layer with normalized predictions.
The mathematical definition of the MLP is rather simple. It takes linear input vector of the form $x_1,...,x_D$ which is multiplied by the weight matrix $\mathbf{w^{(1)}}$ and then transformed using a non-linear activation function $h(\dot$. Due to its simple derivative, mostly the rectified linear unit (ReLU) function is used. This results in the hidden layer, consisting of hidden units. The hidden units get multiplied with the second weight matrix, denoted $\mathbf{w^{(2)}}$ and finally transformed by a sigmoid function $\sigma(\dot)$, which produces the output.Grouping all weight and bias parameter together we get the following equation for the MLP:

\begin{equation}
    y_{k}(\mathbf{x}, \mathbf{w})=\sigma\left(\sum_{j=1}^{M} w_{k j}^{(2)} h\left(\sum_{i=1}^{D} w_{j i}^{(1)} x_{i}+w_{j 0}^{(1)}\right)+w_{k 0}^{(2)}\right)
\end{equation}

for $j=1, \ldots, M$ and $k=1, \ldots, K$, with $M$ being the total number of hidden units and $K$ of the output.


% Application
Since the sigmoid function gives us a probability output, the main function of the MLP is as a classifier. Instead of the initial sigmoid function, it was found to also produce good results for multi label classification transforming the output with a softmax function instead. Images or higher dimensional tensors can be processed by flattening them to a one dimensional tensor. This makes the MLP a flexible and easy to implement model \cite{bishop2006pattern}.

\\
\subsubsection{Graph convolutions}\\
CNNs have shown great results in the field of images classification and object detection. This is due to the fact that a convolution layer takes into account the relation of one pixel to its neighbors. The same holds for graph CNNs where at each convolution the information of each node is passed on as messages to all its neighbors. Each convolution applies an activation function in between the steps. In this case the information is the edges each node has. To process node attributes and edge atributes we have to look at more complex models \cite{tiao_variational_nodate}. 
\\
\subsubsection{RGCN}
Realtional Graph Convolution Net (RGCN) was presented in \cite{kipf_semi-supervised_2017} for edge prediction. This model takes into account features of nodes. Both the adjacency and the feature matrix are matrix-multiplied with the weight matrix and then with them-selves. The resulting vector is a classification of the nodes.
\\
\subsubsection{Graph VAE}
% Encoder options: MLP RGCN
% Decoder MLP
% One shot: creating adjacency and feature matrix at once.
Now we have all the building bocks for a Graph VAE. The encode can either be a MLP, a GCNN or an RGCN. The same holds for the decoder with the addition that model architechture needs to be inverted. An verison of a Graph VAE presented in \cite{simonovsky_graphvae_2018}. This model combines both the previous methods. The input graph undergoes relational graph convolutions before it is flattened and projected into latent space. After applying the reparametrization trick, a simple MLP decoder is used to regenerate the graph. In addition the model concatenates the input with a target vector $y$, which represents ???. The same vector is concatenated with the latent tensor. ***Elborate why they do that***.

Graphs can be generated recursively or in an one-shot approach. This paper uses the second approach and generates the full graph in one go. ***Cite?***
% This model will be the starting point for our research.

\subsubsection{One Shot vs. Recursive}
% One shot: MNIST vs recursive on graphs: Belli
The concept of the VAE has been used to generate data for various usecases. When using the VAE as generator, by sampling from the approximated posterior distribution $q_{\phi}\left(\mathbf{z}$, we can reconstruct the data in a singular run or recursive manner.

The one shot method is the used in the popular example of the VAE generator on the MNIST daataset [cite], as well as on [the faces dataset]. each sample is independent from each other.

Recursive methods take part of the generated datapoint as input for the next datapoint, thus continuously generating the sample. This has been appied to [voice] and to reproduce videogame environments \cite{ha_world_2018}. In \cite{belli_image-conditioned_2019} variation of the GraphVAE has been used to recursively construct a vector roadmap, which has been presented in [link].

For this thesis, we will use the one-shot method, predicting each datapoint independent from each other. The predictions will sparse graph representation with $n$ nodes. A single triple being $n=2$ and a subgraph representation $2<n<100$. [TODO]


\subsection{Graph Matching}
Intro to graph matching on sparse graphs.

\subsubsection{Permutation Invariance}

% Permutation Invariance
% The position or rotation of a graph can vary. 
% Use graph matching to detect similarities between graphs

Permutation invariance refers to the invariance of a permutation of an object. An visual example is the the image generation of numbers. If the loss function of the model would not be permutation invariant, the generated image could show a perfect replica of the input number but due to positional permutation the loss function would penalize the model. 
OR: An example is in object detection in images. An object can have geometrical permutations such as translation, scale or rotation, none the less the model should be able to detect and classify it. In that case, the model is not limited by permutations and  is there fore permutation invariant.
In our case the object is a graph and the nodes can take different positions in the adjacency matrix. To detect similarities between graphs we apply graph matching.

\subsubsection{Graph matching algorithms}
These are three of the state of the art graph matching algorithms.

\begin{itemize}
    \item Wasserstein
    \item Maxpooling
    \item one more
\end{itemize}

% present different ones or only max-pooling?
There are various graph matching algorithms. The one we will implement is the max-pooling (Finding Matches in a Haystack: A Max-Pooling Strategy for Graph Matching in the Presence of Outliers). 

% Max-pooling algorithm comes here !!!
The max-pooling graph matching algorithm returns a symmetric affinity matrix for all nodes

The resulting similarity matrix gives us $X*$ which is continuous and therefore useless. To transform is to a discrete $X$ we use the hungarian algorithm (GPU-accelerated Hungarian algorithms for the Linear
Assignment Problem)

\textbf{Hungarian algorithm}\\
The hungarian algorithm is used to find the shortest path within a matrix. This could be the most efficient work-distribution in a cost matrix. Or \dots \\
It consists of four steps of which the last two are repeated until convergence. This algorithm is not scalable. The Munks algorithm (reference) tackles this problem by?? and is scalable.

second option: \leavevmode
Compare only graph structure. 
NX algorithms: Greedy, Shortest path \dots
\\
\subsubsection{Graph Matching Loss}
The loss is discribed in \cite{simonovsky_graphvae_2018} as.
% \begin{align*}
%     -\log p(G \mid \mathbf{z}) &=-\lambda_{A} \log p\left(A^{\prime} \mid \mathbf{z}\right)-\lambda_{F} \log p(F \mid \mathbf{z})-\\
%     &-\lambda_{E} \log p(E \mid \mathbf{z})
%     &\log p\left(A^{\prime} \mid \mathbf{z}\right)=\\
%     &=1 / k \sum_{a} A_{a, a}^{\prime} \log \widetilde{A}_{a, a}+\left(1-A_{a, a}^{\prime}\right) \log \left(1-\widetilde{A}_{a, a}\right)+\\
%     &+1 / k(k-1) \sum_{a \neq b} A_{a, b}^{\prime} \log \widetilde{A}_{a, b}+\left(1-A_{a, b}^{\prime}\right) \log \left(1-\widetilde{A}_{a, b}\right)\\
%     &\log p(F \mid \mathbf{z})=1 / n \sum_{i} \log F_{i, \cdot}^{T} \tilde{F}_{i,}^{\prime}\\
%     &\log p(E \mid \mathbf{z})=1 /\left(\|A\|_{1}-n\right) \sum_{i \neq j} \log E_{i, j}^{T}, \widetilde{E}_{i, j, \cdot}^{\prime}
% \end{align*}

In contrast to his implementation we assume, that a node or edge can have none, one or multiple attributes. Therefore our attributes are also not sigmoided and do not sum up to one. This leads o the modification of term logF and logE where we do not matrix multiply over the attribute vector but take the BCE as over the rest of the points.
KG can have multiple or no attributes vs molecular graphs can be one hot encoded.

Okay, further we need to treat the $log_pE$ and $log_pF$ just like $log_pA$ and subtract the inverse. Otherwise the model learn to predict very high values only. 

A note to the node features, these stay softmaxed and one-hot encoded since we will use them as node labels.

\subsection{Knowledge Graphs}

Knowledge Graphs are great! The best in the world.
% Knowledge Graphs which we will be using
% We will focus on the generation of KGs.
% Representation of KG as adjacency, edge feature and node feature matrix

% What is a KG


% Hirarchy, enteties, classses


% Othollogy - semantics


% Sparse and dense representation


% Usecases of KG


% Knowledge graphs have very different formats. The datasets we will be sorting with are in rdf format.
% This format can include an defined onthology or not.
% This means the KG consists of triples subject, relation, object.
% when indexing these triples. we get a dense representation of the KG.
% About sparse KGs

\subsection{Ranger Optimizer}
A optimizer, which manged to improve results upon state of the art models. It combines rectified adam, lookahead and gradient centralization.
HOW ABOUT A CITATION


\section{Methods}
This section describes the methods used in the experiments of this thesis. We will begin with data formatting and preprocessing, then the presentation the model, mainly the encoder and decoder and their variations. Moving on we explain our implementation of graph matching the loss function of the model and its evaluation metrics. To end of this chapter we describe the link prediction pipeline which is the main experiment to evaluate our model. All experiments are aimed to be fully reproducible and the model meant to be used in future work, thus the code is openly available on Github \footnote{\url{https://github.com/INDElab/rgvae}}.

\subsection{Knowledge graph data}
All our presented methods will operate on KG data. While data from other graph domains is possible, this work focuses solely on datasets in triple format. We will explain the sparse graph representation, which is the input format for our model and how to preprocess the original KG triples to match that format.

\subsubsection{Sparse representation}

% The first step in our pipeline is the representation of the KG in tensor format. In order to represent the graph structure we use an adjacency matrix $A$ of shape $n\times n$ with $n$ being the number of nodes in our graph. The edge attribute or directed relations between the nodes are represented in the matrix $E$ of shape $n\times n\times d_E$ with $d_E$ being the number of edge attributes. Similarly for node attributes we have the matrix $F$ of shape $n\times d_N$ with $d_N$ number of node attributes. The input graph can have less nodes than the maximum $n$ but not more. The diagonal of the adjacency matrix is filled with $1$ if the indexed node exists, and with $0$ otherwise. The number and encoding of the attributes must be predefined and cannot be changed after training. This way we can uniquely represent a KG.

In this work, we opt for the sparse graph representation $G(A,E,F)$, where $A$ denotes the adjacency matrix, $E$ the edge feature matrix and $F$ the node feature matrix. This allows models architectures as discussed in \ref{ssec:GVAE}. The graph is binary and each matrix is stored in a separate tensor.

% adjacency matrix
The adjacency matrix $A$ takes the shape $n\times n$ with $n$ being the number of nodes in our graph/subgraph. While most of previous work would only allow edges on the upper triangular adjacency matrix and fill the diagonal with ones, we chose a less constrained representation, which we assume is a better fit for representing KGs. In particular, we allow self-loops, meaning a triple where object and subject are the same entity and our relations are directed and can be inverted. Thus $A$ can have a positive signal at any position $A_{i,j}$  $i,j \in \mathbb{R}^{n \times n}$, indicating a directed edge between node of index $i$ and node of index $j$, while $A_{i,j}$ differs from $A_{j,i}$.

% edge attribute matrix
The edge attribute matrix $E$ takes the shape $n\times n\times n_e$ with $n_e$ being the number of unique entities in our dataset. For each possible edge in the adjacency matrix we have a one hot encoded vector pointing to the unique relation in the dataset. Stacking these vectors leads to the three dimensional matrix $E$.

% node attribute matrix
The shape of node attributes matrix $F$ is $n\times n_e$ with $n_e$ being the number of node attributes describing the nodes. Considering that we will split the KG in subgraphs, we use the entity index as node attribute, making it possible to assign every node in a subgraph to the entity in the full KG. Thus, the number of node attributes $n_e$ equals the number unique entities in our dataset. Again the node attributes are one hot encoded vectors, which stacked result in the two dimensional $F$ matrix.


\subsubsection{Preprocessing}
Our datasets consist of three tab separated value files full of triples for training, evaluation and final prediction.The preprocessing steps do not only account for generating subgraphs in the right format but also to ensure valuable research by withhold the triples in the test set until the final run.

% set of entities, set of relations 
% triple to graph
% all triples
% true triples dict
% indexing
From all three sets, we create a set of all occurring entities and similar set for the relations. Now we can define our dimensions $n_e$ and $n_r$. For both sets we create two dictionaries \textit{index-2-entity} and \textit{entity-2-index}  which map back and forth between numerical index and the string representation of the entity (similar for the relation set). These dictionaries are used to create a train and test set of triples with numeric indices. Depending on if we are in the final testing stage or not we include all triples from the training and evaluation file in the training set and use the triples in the testing file as test set, or we ignore the triples in the test file and use the evaluation file triples as test set.

%  truetriples dict 
Further we create two dictionaries, \textit{head} and \textit{tail} which for all occurring subject and relation combination, contain all entities which would complete it to a real triple in our dataset (similar for all relation and object combinations). This will allows us to filter true triples, an important part of link prediction and helpful in graph generation.  

%  triple 2 graph
The final step of preprocessing is a function, which takes a batch of numerical triples and converts them to a batch of binary, multidimensional tensors $A$, $E$ and $F$. While this might sound easy for only one triple per graph, it proves more complex for graphs with $n>2$ facing exemption cases such as self loops or an entity occurring in two triples. We solve this by creating a separate set for head and tail entities, then storing the indices od both in a list, starting with the subject set and finally using this list as keys for a dictionaries with values in the range to $n$. In both edge cases, this results in an adjacency matrix with a rank lower than $n$. A similar approach, with less edge cases to consider, is used to apply the inverted translation from tensor matrix to triple.

% Graph embeddings? unsupervised approach

\subsection{RGVAE}
The principle of a graph VAE has been explained in \ref{ssec:GVAE}, what also covers the foundation of our model, the Relational Graph VAE (RGVAE). Therefore we will focus on the implementation as well as parameter and hyperparameter choice. Since this work is meant to be a prove of concept rather than aimed at outperforming the state of the art, our model is kept as simple as possible and only as complex as necessary. Our appr
For the encoder we implemented two variations, a fully connected and a convolutional, while for the decoder we opted for a single fully connected network.
%  just explain the code

\subsubsection{Initialization}

The RGVAE is initialized with a set of hyperparameter, which define the input shape. Table \ref{tab:RGVAE} shows a complete list of those parameters and their default values. It is left to mention that we use the Xavier uniform method with a gain of $0,01$ to initialize the weight parameter.

\begin{table}[H]
\centering
    \begin{tabular}{|l|l|l|}
    \hline
    \rowcolor[HTML]{EFEFEF}
    \multicolumn{1}{|c}{\textsc{Hyerp.}} & \multicolumn{1}{c}{\textsc{Default}} & \multicolumn{1}{c|}{\textsc{Description}} \\\hline
    $n$     & \multicolumn{1}{c|}{$2$} & Number of nodes  \\
    $n_e$   &\multicolumn{1}{c|}{-}   & Total number of entities\\
    $n_r$   &\multicolumn{1}{c|}{-} & Total number of relations\\
    $d_z$ &\multicolumn{1}{c|}{$100$}   & Latent space dimension\\
    $d_h$ &\multicolumn{1}{c|}{$1024$}   & Hidden dimension\\
    $dropout$ &\multicolumn{1}{c|}{$0.2$}   & Dropout\\
    $\beta$ & \multicolumn{1}{c|}{$1$}  & $\beta$ value for regularization  \\
    $perminv$ & \multicolumn{1}{c|}{\textbf{True}}  & Permutation invariant loss function  \\
    $clipgrad$ & \multicolumn{1}{c|}{\textbf{True}}  & Learning w/ gradient clipping  \\
    $encoder$ & \multicolumn{1}{c|}{\textbf{MLP}}  & Learning w/ gradient clipping  \\
    \hline
    \end{tabular}
\label{tab:RGVAE}
\caption{The inital hyperparameter of the RGVAE with default value and description.}
\end{table}


\subsubsection{Encoder}
% \\Convolution part
% \\RCGN relation Convolution neural net
% \\MLP encoder
% \\Latent space
% \\reparametrization trick
% \\MLP decoder
% \\Graph matching
% \\Discretization of prediction

% MLP
The prove-of-concept encoder is a MLP as described in \ref{ssec:mlp}, which takes the flattened concatenated threefold graph $x=G(A,E,F)$ as batch input. We use the initial parameters to calculate the input

\begin{equation}
\label{eq4:inputdim}
    d_{input} = n*n + n*n*n_r + n*n_e
\end{equation}

The main encoder architecture is a 3 layer fully connected network, with both layers using ReLU as as activation function. The choice for two hidden layers is based on the huge difference between $d_{input}$ and $d_z$. The first layer has a dimension of $2*d_h$ and the option to use dropout, which by default is set to $0.2$. The second (hidden) layer has the dimension $d_h$ which is by default set to $1024$. After the second ReLU activation, the encoder linearly transforms the hidden state to an output vector of $2 \times d_z$. This vector is split and makes the mean and log-variance of size $d_z$ for the reparametrization trick. Sampling $\epsilon$ from an autonomous module, we get the latent representation $z$ of $x$,  the final output of the encoder.

% % Convolutions
The second option for our RGVAE encoder is a GCN as described in \ref{ssec:gcn}. We adopt the architecture from \cite{kipf_semi-supervised_2017} namely two layers of graph convolutions with dropout in between. To match the encoder output to the base model, we then add a flattening and a final linear transformation layer and to substitute the feature matrix used in Kipf's work, we reduce the edge attribute matrix $E$ by one dimension and concatenate it with $F$ resulting in $x_{GCN} \in \mathbb{R}^{n \times (d_e+n*d_r)}$. We forward pass the adjacency matrix $A$ and $x_{GCN}$ through the first GCN layer with a hidden dimension of $d_h$ and ReLU as activation function, followed by a dropout layer. It should be mentioned that dropout is only applied during learning, not on evaluation. The second GCN layer takes the hidden state and again $A$ as input the two dimensional output from the previous layer. Now, instead of having the GCN predict on a number of classes, we have it output a logits vector of dimension $2*d_z$. Therefore we pass the GCN output through a flattening and a linear transformation layer. Similar to above described encoder we use the reparametrization trick to output the latent reparametrization $z$.  


%  TODO:
Table comparing both architectures!!!


\subsubsection{Decoder}


For our RGVAE decoder, we use the same minimal approach as in \cite{simonovsky_graphvae_2018}, namely an MLP with inverted dimensions. The decoder architecture is similar to the one described in  COMPARING TABLE for the MLP encoder version. Since we are decoding the latent space, the input dimension is $d_z$ and the output dimension is $d_{input}$ as calculated in equation \ref{eq4:inputdim}. The flat logits output tensor is split threefold and reshaped to the original input shape of $G(A,E,F)$.   



To sample from the generated graph we apply the Sigmoid activation function to the logits of the first matrix and use the normalized output as weights for binomial distributions, from which we can sample the discrete $\tilde{A}$. For $\tilde{E}$ and $\tilde{F}$ we take the argmax on the last dimension of both matrices, Each node and edge can have only one attribute, referring to its index in $\mathcal{E}$ and $\mathcal{V}$, thus only the highest predicted value is relevant. The generated sample is a discrete graph $\tilde{G}(\tilde{A},\tilde{E},\tilde{F})$.


\subsubsection{Model Limitations}

The main limitation of the RGVAE is the parabolic increase of model parameters with the increase of nodes per input graph $\mathcal{O}(n^2)$. The number of parameters to train is directly linked with the GPU memory requirements. Even more computationally expensive is the use of permutation invariant graph matching, with a complexity of $\mathcal{O}(n^3)$. Thus, we propose this model only for generating small graphs with $n<30$.


% The proposed model is expected to be useful
% only for generating small graphs. This is due to growth
% of GPU memory requirements and number of parameters
% (O(k
% 2
% )) as well as matching complexity (O(k
% 4
% )), with small
% decrease in quality for high values of k. I
% \cite{simonovsky_graphvae_2018}


% \subsubsection{Hyperparameters}
% learning rate
% beta for regularization term
% hidden dimensions
% latent dimensions
% dropout
% n, n_e, n_r
% Convolutions vs no convolutions
% IDEA: Convolutions + MLP


\subsection{RGVAE learning}
% Lay out pipeline.
GPU requirements.
LISA
Nvidia titan RX 25GB 60h
Experiment log wndb.ai [cite]
optimizer ranger github repo link

\subsubsection{Max pooling graph matching}

%  explain the code
% batch implementation
% loop implementation to check
Summing over the neighbors means summing over the whole column
Normalize matrix with Frobenius Norm

Batch version:
Only matmul and dot. keep dimension of S with shape (bs,n,n,k,k)
When maxpooling, flatten Xs (n,k) for batch dot multiplication. This way (i think) we sum over all j nad b neighbors instead of taking the max.  

\subsubsection{Loss function}

% If loss function should be permutation invariant we need to do some kind of graph matching.\\
% Different options for graph matching.\\
% Maxpooling-algorithm:\\
% Assumptions\\
% Node to edge affinity equals 0\\

% Self-loops are possible, adjacency matrix can be zero or one.\\


In contrast to his implementation we assume, that a node or edge can have none, one or multiple attributes. Therefore our attributes are also not sigmoided and do not sum up to one. This leads o the modification of term logF and logE where we do not matrix multiply over the attribute vector but take the BCE as over the rest of the points.
KG can have multiple or no attributes vs molecular graphs can be one hot encoded.

Okay, further we need to treat the $log_pE$ and $log_pF$ just like $log_pA$ and subtract the inverse. Otherwise the model learn to predict very high values only. 

A note to the node features, these stay softmaxed and one-hot encoded since we will use them as node labels.

beta VAE

Hungarian algorithm for discrimination of X


The hungarian algorithm as presented in section 2 return the shortest path in a matrix. We use this shortest path as bast match between the two graphs. The node paris identified as optimal are masked as $1$ and the rest of the matrix as $0$. This way we discretizice $X\star$ to $X$. 


\subsection{Link prediction and Metrics}


% First:
% Link prediction to make a proof of concept, not achieve SOTA.\\
% Node classifier by only using encoder.\\
% latent space interpolation to find analogies to smile vector in the latent space of a face VAEs.\\
% Identify if VAE learns semantics. OWL, onthology datasetbatch_size, required.\\
% Wasserstein distance???
% Link prediction?
% blabl


\begin{itemize}
    \item Load Dataset
    \item Convert to sparse in batches
    \item forward pass through VAE
    \item MPGM loss
    \item Backward pass
    \item test on val set
    \item calculate MRR on subset of val set
    \item draw graphs
\end{itemize}


\subsection{Variational DistMult}

Control model
reparametrization trick in middle
still same scoring function


\label{sec:mthods}

\section{Experiments \& Results}
% Experiments

This section presents the experiments and results aimed at evaluating our proposed graph generative model. First we run a grid-search on the hyperparameter space to find the optimal configuration of the RGVAE. We use the ELBO and MRR as evaluation metric. The best configurations are used to perform link prediction. Here we compare the model performance with MLP versus GCN encoder. We use the VDistMult as control model for link prediction. Finally we run two proof-of-concept experiments. The first generating triples and filter on a entity class constraining relation, thus we get an insight of how much percent of the generated triples are valid. Secondly we analyze the results of a RGVAE trained on subgraphs with $n=10$. For the experiments we use two multi-relational KG datasets.

% We covered link and node prediction and compared those to SOTA scores. Further we ran experiments on investigating the coherence of the reproduced graph structure. Lastly we measured the adherence of our model to the KG's underlying syntax.


\subsection{Data}
\label{ssec5:data}
For this sake of comparison with state of the art results, we chose the two most popular dataset used in this field of KG link prediction, FB15k237 and WN18rr.

% Training models on each dataset for 333 epochs, without early stopping.

\textbf{FB15K-237} is a successor of the FB15K dataset, first introduced by \cite{bordes_translating_2013}, which suffered of major test leakage, meaning that triples from test set could be inferred by inverting triples from the train set. In FB15K-237, introduced in \cite{toutanova_representing_2015} these triples where removed.
The data was scraped from Freebase, while only the most frequent entities and relations were considered. The huge open-world KG Freebase \cite{bollacker_freebase_2008}, which before its discontinuation had around $1.2$ billion triples and $80$ million entities, was structured by assigning types and classes to entities and type constrains to relations. Thus, a triple can only be formed if the relation constrain matches the entity's type. Freebase was free for everyone to access and expand. This led to inconsistencies, duplicates and highly inconsistent notation, which might have been the reason for its discontinuation. Data dumps of the latest version are still available. Since the launch date of Freebase, advanced systems for storage and query of large-scale maintenance KGs have evolved \cite{cudre2013nosql}.






% textwidth for figures:
% \printinunitsof{in}\prntlen{\textwidth}

% linewidth for figures:
% \printinunitsof{in}\prntlen{\linewidth}

\textbf{WN18RR}, a dataset of synonyms and hypernyms, is a successor of yet another dataset WN18 introduced by \cite{bordes_translating_2013}. Similar to the the above the original dataset suffered from test leakage, thus, an updated version without reciprocal triples was introduced by \cite{dettmers_convolutional_2018}. This dataset is characterized by its few relations and large corpus of entities. In contrary to FB15K-237, its triples are difficult to be judged on view and the underlying semantics are entity specific, meaning not relying on types and classes.



\begin{table}[H]
  \centering
      \begin{tabular}{|l|l|l|l|}
      \hline
      \rowcolor[HTML]{EFEFEF}
      \multicolumn{1}{|c}{\textsc{Dataset}} & \multicolumn{1}{c}{\textsc{Entities}} & \multicolumn{1}{c}{\textsc{Relations}} & \multicolumn{1}{c|}{\textsc{Triples}}\\\hline
      FB15K-237     & \multicolumn{1}{c|}{$14,951$} & \multicolumn{1}{c|}{$237$} & \multicolumn{1}{c|}{$310,116$}\\
      WN18RR   & \multicolumn{1}{c|}{$40,943$} & \multicolumn{1}{c|}{$11$} & \multicolumn{1}{c|}{$93,003$} \\
      \hline
      \end{tabular}
      \caption{Statistics of the FB15K-237 \cite{toutanova_representing_2015} and WN18RR \cite{dettmers_convolutional_2018} datasets.}
      \label{tab5:data}
  \end{table}


\subsection{Hyperparameter Tuning}

In this section we run a grid search for the three hyperparameter $\beta$, $d_z$ and $d_h$ for a set of contrastive values. To reduce the computational expenses we train each model for $60$ epochs and  evaluate link prediction on a subset of $50$ triples.

Empirically we set the learning rate to $3e^{-5}$ and the maximum batch size fitting on the GPU memory. For $d_h$ we did not see any significant changes for higher values, thus we choose a lower number to reduce the total model parameters. The remaining hyperparameter did influence and the optimal setting vary for each dataset, table \ref{tab:RGVAEhyp} shows the results of our hyperparameter tuning.

% lr empirically and batchszize fixed.

% beta
For the hyperparameter tuning of $\beta \in [0,1,10,100]$ we chose significant values. With $\beta = 0$ we do not constrain our model on the Gaussian prior, thus the latent distribution can take the form of any distribution. This reduces the influence of the variational module and the model becomes closer to an autoencoder. For $\beta = 1$ we get our base model, and for $\beta \in [10,100]$ the $\beta-$VAE version.


\begin{figure}[H]
    \centering
    \begin{subfigure}{.5\textwidth}
      \centering
      \includegraphics[width=.9\linewidth, keepaspectratio]{graphs/plots/beta_loss_fb.png}
      \caption{FB15K-237}
      \label{fig5:betafb}
    \end{subfigure}%
    \begin{subfigure}{.5\textwidth}
      \centering
      \includegraphics[width=.9\linewidth, keepaspectratio]{graphs/plots/beta_loss_wn.png}
      \caption{WN18RR}
      \label{fig5:betawn}
    \end{subfigure}
    \caption{Validation loss for RGVAE with $\beta \in [0,1,10,100]$ trained on each dataset.}
    \label{fig5:beta}
\end{figure}


Figure \ref{fig5:beta} shows the validation ELBO for the different $\beta$ values and for both datasets. We notice two interesting outcomes.  

\begin{itemize}
    \item For $\beta = 0$ converges further than the rest.
    \item The remaining values behave quase identical with $\beta = 100$ performing slightly better. 
\end{itemize}

Since setting $\beta = 0$ would undermine our hypothesis of evaluating variational model, we chose $\beta = 100$ as default for the following experiments. This also compares with the $\beta$ values proposed by Higgins in \cite{higgins_beta-vae_2016} to achieve a factorization of the latent space.

Especially on the FB15k-237 dataset the $\beta = 0$ configuration converges to a much lower ELBO. Thus, we have the trained models perform link-prediction on a $1\%$ subset of the validation set. Figure \ref{fig5:betafbmrr} indicates an inverse correlation between the ELBO and the MRR score. 

% \begin{figure}[H]
%     \centering
%       \includegraphics[width=.45\textwidth]{graphs/plots/beta_mrr_fb.png}
%       \caption{MRR scores for different $\beta$ values on the dataset FB15k-237.}
%       \label{fig5:betafbmrr}
% \end{figure}



\begin{figure}[H]
  \centering
  \begin{subfigure}{.5\textwidth}
    \centering
    \includegraphics[height=.6\textwidth, keepaspectratio]{graphs/plots/beta_mrr_fb.png}
    \caption{FB15K-237}
    \label{fig5:betamrrfb}
  \end{subfigure}%
  \begin{subfigure}{.5\textwidth}
    \centering
    \includegraphics[height=.6\textwidth, keepaspectratio]{graphs/plots/beta_mrr_wn.png}
    \caption{WN18RR}
    \label{fig5:betamrrwn}
  \end{subfigure}
  \caption{MRR scores during training for different $\beta$ values on $1\%$ of the validation set.}
  \label{fig5:betafbmrr}
\end{figure}

% Barplot IN APPENDIX  

% d_z
Experiments on the impact of $d_z$ on the ELBO show little improvement for $10<d_z<100$ and from $100<d_z<1000$ insignificant to no improvement. Thus, we chose $d_z=100$ as default for our experiments.


% d_h did not influcence
Lastly, we evaluate the models hidden dimensions $d_h$ and its influence on the ELBO and the (subset)MRR. We compare between $d_h\in [256, 512, 1024, 2048]$, while the lowest configuration performs slightly worse on the ELBO, there is no significant difference between the remaining three configurations. Considering the models parameter count we chose the $d_h=512$ as default.


\subsection{Link Prediction}


 We now get to the most extensive experiment of this thesis. The results of this experiment show, if the RGVAE architecture is suitable for link prediction. specifically, if it is able to grasp the underlying semantics of the KG data at significantly better than the baseline of random predictions for differentiating between real and corrupted triples. First we evaluate the performance of the RGVAE on this experiment, comparing both encoder versions. Then we investigate the influence of the variational inference by comparing the variational and original versions of DistMult on link prediction.
 
 \subsubsection{RGVAE}

 At this point we bring in the convolutional variation of our model, which we denote as RGCVAE. The experiments reveal if the convolutional architecture holds an advantage compared to the simple MLP baseline. Further a randomly initiated and untrained RGVAE is used as control model.
 
 Due to its sparse graph computation, the RGVAE takes about 7 days to evaluate link prediction on the full test set and even 3 days when prediction tasks run parallel on a node with $4$ GPUs. Since the exemplary link prediction during experimenting with different hyperparameter already gave us an idea of the mediocre performance of our model, we chose to spare computation time and power by running link prediction on a randomly drawn one-third of the complete test set. Each run is repeated three times using a different random seed.

The results are visualized in figure \ref{fig5:lp_final}. We chose a visualization over a table, to emphasize the observed differences between encoders and datasets. For the reason of the mentioned random subset of the test set, are these results not suitable for academically valid comparisons with models trained on the full testset. Note that the figures are scales to a range $[0,0.1]$ while all metrics have a maximum of $1$. The dotted line represents the baseline score of an untrained RGVAE with MLP encoder.

Comparing a MRR of $0.08$ to the DistMult score or $0.3$ our model does not perform competitively on link prediction tasks.

% Better than random?? I hope so.

Graph convolutions do not yield an advantage over the MLP encoder. In fact, the RGVAE with GCN encoder even scores slightly worse. 

%  FOR Conclusion: this might be because of the implementation of stacking the matrices.
The model scores about three times better on the FB15K-237 dataset than on WN18RR. FB15k-237 is a richer dataset with more triples and and a more balanced ratio of entities to relations. WN18RR operates on only 18 relations, what makes the relation most crucial when completing a triple. The architecture of the RGVAE puts twice the emphasis entities, described by the adjacency and the node feature matrices, while the relation is only represented by the overly sparse edge attribute matrix. Thus, we could conclude that our model learns to predict based on the hidden types and topics of the entities. All possible conclusions for this are discussed in chapter \ref{sec:discus}. Relevant for this section is solely, that we chose the FB15K-237 dataset to investigate further, how well the RGVAE's grasps the underlying entity types and triple topics.   

%  REASON: fb has more relations, is a more complete KG. wn only 12 relations and more entities. Model emphasizes entities (adj and node features), relations may be less relevant, wn is more about the relation (same/ not the same). This indicates that our model grasps the hidden types and topics of the FB entities.


 \begin{figure}[H]
  \begin{subfigure}{.5\textwidth}
    \left
    \includegraphics[height=.5\textwidth, keepaspectratio]{graphs/plots/lp_fb.png}
    \caption{FB15K-237}
    \label{fig5:lpfb}
  \end{subfigure}%
  \begin{subfigure}{.5\textwidth}
    \right
    \includegraphics[height=.5\textwidth, keepaspectratio]{graphs/plots/lp_wn_wol.png}
    \caption{WN18RR}
    \label{fig5:lpwn}
  \end{subfigure}
  \caption{Link prediction scores of the RGVAE. Dotted line represents MRR baseline of an untrained model. }
  \label{fig5:lp_final}
\end{figure}

% TODO add random!!!

% Compare with vs without convolution 
% We use negative elbo as scoring function. Since elbo is aimed to be reduced and LP scores are higher better.

% We try with and without permutation

% We try the model as encoder only NO

% We use 1/3 of the test set only, randomly drawn. Run 3 times?
% Final models only 60 epochs


\subsubsection{Impact of Variational Inference and Gaussian prior}

In order to explain the poor performance of the RGVAE on the task of link prediction, we investigate the impact of the variational inference. Since the RGVAE with relaxed latent space, meaning less variance, indicated higher scores than the version with Gaussian prior, we examine the two variants by means of embedding models. The original DistMult model with optimized parameter serves as control model, while we compare it to the VDistmult, described in section \ref{ssec4:vdistm}, learning the full ELBO versus learning only on the reconstruction loss. By not including the regularization term in the loss the model is no longer bound to the Gaussian prior, which results in a relaxation of the latent space.

We train the three models for $300$ epochs solely on the FB15k-237 dataset and evaluate MRR, Hits@$1$, Hits@$3$ and Hits@$10$. Table \ref{tab5:VarDistM} shows the mean scores with $\mu \pm \sigma$ of three runs per model. Note that the exponent on $\sigma$ holds for the whole term. We can clearly see that both variational versions of the DistMult perform significantly worse than the original model. Learning on the full elbo or only the reconstruction loss does not seem to influence the scores in this setting. This indicates, that the models performance on link prediction suffers from using variational inference.

In the last row we show the results of the RGVAE with relaxed latent space. This model was trained with $\beta=0$ thus not constraining the latent space on a Gaussian prior. The model outperforms the versions with hyperparameter choice $\beta>0$ and scores the closest to the DistMult model. The impact of $\beta$ shows in the regularization, which we tracked separately. The maximum values of$D_{K L}$ during the experiment are 

\begin{equation}
  \begin{align}
    D_{reg} &= \beta D_{K L}\left(q_{\phi}\left(\mathbf{z} \mid G\right) \| p_{\theta}(\mathbf{z})\right) \\
    \max_{\beta = 0} D_{K L} &= 3506 \\
    \max_{\beta = 100} D_{K L} &= 0.0154
  \end{align}
  \label{eq5:KLdifferentBeta}
\end{equation}


While the results of the relaxed RGVAE might seem promising, Distmult is a much simpler and faster link predictor, thus we do not see a justification to keep researching on the RGVAE for this task. Note that due to the high computation cost of the RGVAE we only run the experiment once on the full dataset.

% TODO: Answer question:Link prediction with control model:

% Trained for 300 epochs

% We see that the variational part messes everything up.

% Table:
% MRR + Hits@all + Loss

\begin{table}[H]
  \centering
      \begin{tabular}{|l|l|l|l|l|}
      \hline
      \rowcolor[HTML]{EFEFEF}
      \multicolumn{1}{|c}{\textsc{Model}} & \multicolumn{1}{c}{\textsc{MRR}} & \multicolumn{1}{c}{\textsc{Hits@$1$}} & \multicolumn{1}{c}{\textsc{Hits@$3$}} & \multicolumn{1}{c|}{\textsc{Hits@$3$}} \\\hline
      DistMult     & \multicolumn{1}{c|}{$0.2854\pm 0.0025$} & \multicolumn{1}{c|}{$0.2\pm 0.001$} & \multicolumn{1}{c|}{$0.3149\pm 0.0038$} & \multicolumn{1}{c|}{$0.4512\pm 0.0053$}  \\
      VDistMult   & \multicolumn{1}{c|}{$0.517\pm 0.0197e^{-3}$} & \multicolumn{1}{c|}{$0.2442\pm 0.1994e^{-4}$} & \multicolumn{1}{c|}{$0.8145 \pm 0.3049e^{-4}$} & \multicolumn{1}{c|}{$0.399\pm 0.0576e^{-3}$} \\
      VDistMult w/ ELBO   & \multicolumn{1}{c|}{$0.6397\pm 0.0357e^{-3}$} & \multicolumn{1}{c|}{$0.57\pm 0.3046e^{-4}$} & \multicolumn{1}{c|}{$0.1547\pm 0.1023e^{-3}$} & \multicolumn{1}{c|}{$0.6351\pm 0.1992e^{-4}$} \\
      RGVAE w/o ELBO   & \multicolumn{1}{c|}{$0.1412$} & \multicolumn{1}{c|}{$0.0981$} & \multicolumn{1}{c|}{$0.1494$} & \multicolumn{1}{c|}{$0.2275$} \\
      \hline
      \end{tabular}
      \caption{Link prediction scores of DistMult and RGVAE versions on the FB15k-237 dataset.}
      \label{tab5:VarDistM}
  \end{table}


\subsection{Impact of permutation}
% Check if adj matrix adheres to edge attribute matrix.

Furthermore we examine the influence of the permutation invariant loss function described in \ref{ssec4:loss}. During training and subset link prediction no significant difference was observed between the RGVAE with versus without matching target and prediction graph. Yet, two observations draw our attention, namely:

\begin{itemize}
  \item The amount of nodes permuted per batch converges during training from $100$\% to exactly $80$%.
  \item The RGVAE with permutation invariant loss function learns to predict many variations of adjacency matrix while the standard model predicts similar to the target.
\end{itemize}

The first point indicates that the model learns a set of adjacency matrices, which can be permuted to match the target while optimizations the loss. Note that the generated matrix representation of the triples either has only one edge on the right upper index $A_{0,n}$ or, in the rare case of self-loops in $A_{0,0}$. The number of nodes per graph for these observations is set to $n=2$. Curiosity remains why the model converges to steadily permute $\frac{3}{5}$ of the prediction.
Secondly, we see that even when converged, the model predicts variations od the adjacency matrix very different to the target. The most common is a single edge on $A_{n,0}$ and on $A_{n,n}$. Less common and with lack of explanation are the predictions of an empty, or multi edge adjacency matrix. In contrast to this and as expected, the RGVAE with the standard loss function learns to solely predict edges on $A_{0.0}$. 
Finally we analyze the impact of permutation invariance on the experiment of generating valid triples section \ref{ssec5:syntax}.


% Permutation starts at $100\%$ at the beginning of training and converges to $60\%$.

% Model without predicts adj node always in the upper right just as the target. Model with predicts much more variations of adjacency.

% Graph of permutation during training.
% loss with vs without 


\begin{figure}
  \right
  \begin{subfigure}{.55\textwidth}
    \left
    \includegraphics[height=.5\textwidth, keepaspectratio]{graphs/plots/permute_loss.png}
    \label{fig5:permELBO}
  \end{subfigure}%
  \begin{subfigure}{.55\textwidth}
    \left
    \includegraphics[height=.5\textwidth]{graphs/plots/permute_permutation_wol.png}
    \label{fig5:permRate}
  \end{subfigure}
  \caption{RGVAE validation loss (a) and rate of permuted nodes (b) during training.}
  \label{fig5:permInv}
\end{figure}

\subsection{Interpolate Latent Space}

Inspired by the popular results of Higgins, who featurizes each latent dimension on a facial features of the FACES dataset \cite{ebner_facesdatabase_2010}. The VAE generates faces controlling feature such as age, gender and emotions by manipulating single latent dimensions \cite{higgins_beta-vae_2016}.  We run this experiment with the RGVAE on the FB15k-237 dataset, using two different interpolation methods. The latent dimension for this experiment is set to $d_{z}=10$ in order to analyze each dimension separately and the interpolation is linear with a step count of $10$.

The first experiment is linear interpolating between two triples. Therefor two valid triples from the train set are encoded into their latent representation. We chose the two semantically related triples triples to analyze if the linear movement in latent space correlates with an obvious semantic feature. 

\begin{center}
  \texttt{[['/m/02mjmr Barack Obama'], ['/people/person/place\_of\_birth'], ['/m/02hrh0\_	Honolulu']]}
  \texttt{[['/m/058w5 Michelangelo'], ['/people/deceased\_person/place\_of\_death'], ['/m/06c62	Rome']]}
\end{center}


% TODO present the results and link to appendix

\begin{table}[H]
  \centering
  \begin{tabular}{|c|}
  \hline
  \rowcolor[HTML]{EFEFEF} 
  \textsc{RGVAE permutation}\\ \hline
  \texttt{[[France] [/base/petbreeds/city\_with\_dogs/top\_breeds] [Imperial Japanese Army]]}\\
  \texttt{[[Cree Summer] [/tv/tv\_program/program\_creator] [David Chase]]}\\
  \texttt{[[Guitar] [/business/business\_operation/assets] [Paramount Vantage]]}\\
  \texttt{[[Democratic Party] [/music/genre/artists] [Howard Hawks]]}\\
  \texttt{[[Roy Haynes] [/people/person/spouse\_s] [The Portrait of a Lady]]}\\
  \texttt{[[Cleveland Browns] [/film/actor/dubbing\_performances] [David Milch]]}\\
  \texttt{[[Jay-Z] [/film/special\_film\_performance\_type/film\_performance\_type] [Ashley Tisdale]]}\\
  \texttt{[[Phoenix Suns] [/film/film/dubbing\_performances] [Lynn]]}\\
  \texttt{[[James E. Sullivan Award] [/organization/organization/child] [Giant Records]]}\\
  \texttt{[[Boston United F.C.] [/soccer/football\_player/current\_team] [Kensal Green Cemetery]]}\\  
  \hline
  \end{tabular}
\caption{Latent space interpolation between two triples in $10$ steps.}
\label{tab5:ipbtw2}
\end{table}
% Obama triple is not reproduced, not even close.

For the second experiment, we interpolate each latent dimension isolated in a $95\%$ confidence interval of the Standard Gaussian distribution. Starting with the encoded representation of the Obama triple, we incrementally add $z_{i} = -1.96 + j \times s$ with $s = \frac{1.96 * 2}{n_s-1}$ for the number of steps $n_s = 10$. Due to the size of the tables representing the interpolations and the low value they add to the presentation of this work, the result of this experiments can be found in the annex \ref{annexB:95}.


% Further we go ahead and test what happens if we modify one latent dimension at a time with $d_z = 10$ of a triple. TABLE: (s,r,o), x axis dims, y axis steps. $95\%$ Gaussian confidence 

% Can the model assign logical features to latent dimensions?


\subsection{Generator Validation}
\label{ssec5:syntax}

On closed-world schema-based KGs the approach for testing the validity of a new triple is to add it to the existing KG and run a onthology reasoner on it. A inconsistency in the KG will appear as \texttt{null}-Class, but only if an axiom is violated. This approach works only for fully constrained KGs and is not scalable, since the reasoner recursively checks every triple for every axiom. Thus, we present an alternative and improvised way to estimate the validity of generated triples.

The FB15K-237 is a subset from the FreeBase KG \cite{bollacker_freebase_2008}, thus, even thought they are not part of the dataset, $14541$ entities have type properties in their original Freebase representation. 
Querying the last official Freebase dump, as proposed by Xie \textit{et al.} \cite{xie2016representation}, we get the types for each entity in the FB15K-237 dataset, With exception of $8$ entities, which could not be found in the query.

Our approach is to randomly generate triples, from signals randomly drawn from a Standard Gaussian distribution. Then to filter those triples on predicates which contain the type \textit{people}, which is within the top 10 most common Freebase types. We differentiate between base-class types subclass types, both can contain the word \textit{people}. The entity \texttt{['/m/02mjmr Barack Obama']} has between many others the type \texttt{[/people/measured\_person]}. Here the base class is \textit{people} and the subclass \textit{measured\_person}. We could filter directly on subclasses, but we chose to give our model more creative freedom and filter for \textit{people} in the full set of types. Using this choice, the generated triples are scored on logic rather than facts. E.g. any person can hypothetically the a \textit{measured\_person}, understanding this implies semantical reasoning, while differentiating between which person is and is not a \textit{measured\_person} implies contextual knowledge. Thus, we use the base type \textit{people} to validate triples.
Furthermore the relations are inconsistent in their notation, partly not only having a head type constrain. Thus, we check only the head entity for the key type. 

\begin{itemize}
  \item From $14541$ entities, $5283$ contain the keyword people, or $36.332$\%.
  \item From $237$ predicates, $25$ contain the keyword people, or $10.549$\%.
  \item From $310116$ triples, $47354$ contain the a predicate keyword people, or $15.269$\%.
\end{itemize}

Considering these facts, we calculate the marginal probability of guessing a head entity $s$ of type \textit{people}, given a triple which contains the type \textit{people}. Without prior knowledge $p(s_{p})$ and $p(r_{p})$ are conditionally independent. The probability is calculated as:

\begin{equation}
  p(s_{p} \mid r_{p}) &= p(s_{p}) = 0.3633
  \label{eq5:randomValid}
\end{equation}

For this experiment we generate triples until $10e^5$ contain the key type. Those filtered triples are validated on the type of the head entity and compared to the full dataset for novelty. Here we again compare the performance between the RGVAE with the two different encoder architectures. Furthermore the models are trained both with regular and permutation invariant loss function. Lastly, the experiments are repeated for sampling the latent signal from $\mathcal{N}(0,1)$ and for sampling from $\mathcal{N}(0,2)$.  We average the accuracy of three runs of generating a valid triple and of this triple being unseen in the dataset. The results for valid triples are shown in figure \ref{fig5:syntax}. The dotted horizontal line indicates the random probability of generating a valid triple, calculated in equation \ref{eq5:randomValid} and $\sigma^2_1$ and $\sigma^2_2$ denote the variance for the different latent space distributions. 



\begin{figure}[H]
  \centering
  \includegraphics[height=.3\textwidth, keepaspectratio]{graphs/plots/kg_all.png}
  \caption{Accuracy of generating valid triples.}
  \label{fig5:syntax}
\end{figure}

To our disappointment the model does not perform significantly better than the baseline of random predictions. Neither the choice loss function nor the doubled variance show a correlation with the accuracy. The only configuration standing out is the RGVAE with convolutional encoder, standard loss function and $\sigma^2=2$, scoring $4$\% higher than the baseline. From all valid generated triples $100\pm 0.001$\% are new and unseen in the dataset. Coming back to \texttt{['/m/02mjmr Barack Obama']}, we filter the unseen triples fo the first three appearances of this entity. These are displayed in table \ref{tab5:genTriples} for every variation of the RGVAE and in the same order as in figure \ref{fig5:syntax} 



\begin{table}[H]
  \begin{tabular}{|c|}
  \hline
  \rowcolor[HTML]{EFEFEF} 
  \textsc{RGVAE standard} $\sigma_2^2$\\ \hline
  \texttt{[[Barack Obama]	[/people/person/places\_lived./people/place\_lived/location]	[Casablanca]]}\\
  \texttt{[[Barack Obama]	[/people/person/place\_of\_birth]	[Sarah Silverman]]}\\
  \texttt{[[Barack Obama]	[/people/person/place\_of\_birth]	[The League of Extraordinary Gentlemen]]}\\ \hline
  \rowcolor[HTML]{EFEFEF} 
  \textsc{RGVAE permuted} $\sigma_2^2$\\ \hline
  \texttt{[[Barack Obama]	[/people/ethnicity/geographic\_distribution]	[End of Watch]]}\\
  \texttt{[[Barack Obama]	[/people/profession/specialization\_of]	[Montgomery County]]}\\
  \texttt{[[Barack Obama]	[/people/cause\_of\_death/people]	[WWE Superstars]]}\\ \hline
  \rowcolor[HTML]{EFEFEF} 
  \textsc{RGVAE standard} $\sigma_1^2$\\ \hline
  \texttt{[[Barack Obama]	[/people/person/place\_of\_birth]	[Academy Award for Best Sound Editing]]}\\
  \texttt{[[Barack Obama]	[/people/person/places\_lived./people/place\_lived/location]	[Stan Lee]]}\\
  \texttt{[[Barack Obama]	[/people/person/place\_of\_birth]	[Multiple sclerosis]]}\\ \hline
  \rowcolor[HTML]{EFEFEF} 
  \textsc{RGVAE permuted} $\sigma_1^2$\\ \hline
  \texttt{[[James Brolin]	[/people/person/places\_lived./people/place\_lived/location]	[Barack Obama]]}\\
  \texttt{[[Barack Obama]	[/people/person/spouse\_s./people/marriage/location]	[D.C. United]]}\\
  \texttt{[[Jim Sheridan]	[/people/person/religion]	[Barack Obama]]}\\ \hline
  \rowcolor[HTML]{EFEFEF} 
  \textsc{cRGVAE standard} $\sigma_2^2$\\ \hline
  \texttt{}\\
  \texttt{None}\\
  \texttt{}\\ \hline
  \rowcolor[HTML]{EFEFEF} 
  \textsc{cRGVAE permuted} $\sigma_2^2$\\ \hline
  \texttt{[[Pinto Colvig]	[/people/deceased\_person/place\_of\_burial]	[Barack Obama]]}\\
  \texttt{[[Helena Bonham Carter]	[/people/person/gender]	[Barack Obama]]}\\
  \texttt{[[John Buscema]	[/people/person/spouse\_s./people/marriage/spouse]	[Barack Obama]]}\\ \hline
  \rowcolor[HTML]{EFEFEF} 
  \textsc{cRGVAE standard} $\sigma_1^2$\\ \hline
  \texttt{[[Suhasini Ratnam]	[/people/person/sibling\_s./people/sibling\_relationship]	[Barack Obama]]}\\
  \texttt{[[Barack Obama]	[/people/person/gender]	[Deva]]}\\
  \texttt{[[Barack Obama]	[/people/person/sibling\_s./people/sibling\_relationship]	[Niagara Falls]]}\\ \hline
  \rowcolor[HTML]{EFEFEF} 
  \textsc{cRGVAE permuted} $\sigma_1^2$\\ \hline
  \texttt{[[Jonathan Rhys Meyers]	[/people/person/nationality]	[Barack Obama]]}\\
  \texttt{[[Barack Obama]	[/people/person/sibling\_s./people/sibling\_relationship]	[Motherwell F.C.]]}\\
  \texttt{[[Pinto Colvig]	[/people/deceased\_person/place\_of\_burial]	[Barack Obama]]}\\ \hline
  \end{tabular}
\caption{Generated and unseen knowledge.}
\label{tab5:genTriples}
\end{table}


The generated triples confirm the accuracy results. With exception of the triple 
\begin{center}
  \texttt{[[Barack Obama]	[/people/person/places\_lived./people/place\_lived/location]	[Casablanca]]} 
\end{center}

all remaining triples violate common sense logic. The RGVAE does not differentiate between the types gender, location, movie, person or medicine. It even goes so far to state that Obama was born in \texttt{[Multiple sclerosis]}. While this might sound funny it also clearly indicates, that our model did not learn the underlying semantics of this real world KG.

While investigating the model and the generated triple set, we notice two outcomes. Neither the augmentation of the variance nor the enabling of the permutation invariance has an impact on either of the model with two encoder versions. The regularization loss converges during training to zero, meaning that the model learns a latent representation of the dataset as  nearly perfect Standard Gaussian distribution. Yet, even when sampling latent signal from the exact same distribution, we notice that each model repeatedly predicts combinations of a small subset of entities and relations, e.g. the RGVAE version which did not predict the Obama entity once in a total of $111583$ valid triples. This also aligns with the interpolation results, where we observed a static relation for the full gridsearch of the latent space. If we look at the gradient and parameter values $\phi$ and$\theta$ of the MLP encoder and decoder, we see a much higher variance and gradients for $\phi$. The decoder shows higher values and variance for a small subset of neighboring parameter, while the remaining parameters converge to a very similar and low value. This indicates that the encoder learns very well to represent each different triple as Standard Gaussian latent representation. The decoder MLP on the opposite seems to to ignore most of this representation by assigning vanishing values to the connected parameters. Intuitively it seems that the decoder learns to interpret the part of the latent representation corresponding to the adjacency matrix and minimizes as well as stabilizes the loss of the edge and node attribute matrix by uniformly distributing their probability. This leads to the decoder reconstructing edge and node attribute randomly. Further, depending on the values of $\theta$ when the model finishes learning, the decoder repeatedly predicts the same subset of entities and relation independent of the latent signal $z$. If we look at the flattened representation of our input graph, we see that the part representing the adjacency matrix is way shorter and has a tractable mean of $\frac{1}{4}$ while the mean for the edge and node attribute matrices are $\frac{1}{{4 \times 1345}}$ and $\frac{1}{14951}$. The problem of a our decoder partly ignoring the latent input and potential solutions are discussed further in section \ref{ssec7:collapse}. 


\subsection{Delta Correction}
\label{ssec5:delta}

% Explain quick delta implementation 
A solution to the observed phenomenon, which is known as decoder collapse, is the use of a $\delta$ parameter, presented by Razavi \textit{et al.} in \cite{razavi_preventing_2018}. The regularization term is expanded by subtracting the $\delta$ parameter and by taking the absolute value. This forces the latent space into a truncated Gaussian distribution. The simple nature of this solution allows us to integrate it in the RGVAE setup and evaluate its impact on the previous experiments.

% Explain new experiments
In a finall experiment the RGVAE is trained in $4$ different modes, with $\delta \in [0, 0.6]$ and the using the standard versus the graph matching loss function. Since we do not expect this new parameter to have a significant impact on the link prediction results, the most informative and comparable experiments are latent space interpolation and the generation of valid triples.

% show interesting interpolation
Table \ref{tab5:ipbtw2Delta} shows the results for $\delta = 0.6$ and graph matching. The model shows the same behavior as the for $\delta = 0$. The start and end triple do not match the encoded target triples and $9$ out of $10$ interpolation steps predict the same relation. A difference can be seen in the traversing of each latent dimension. Here minimal semantic coherence can be observed \textbf{(Table in Appendix)}.
The new parameter $\delta$ did not raise the ratio of valid generated triples above the baseline.
A difference between standard, appended in the the annex \ref{annexA:ipbtw2DeltaNoPerm}, and graph matching loss could also not be observed. The results of the full latent space interpolation for both variations of loss functions are appended in the annex \ref{annexB:95}.

\begin{table}[H]
  \centering
  \begin{tabular}{|c|}
  \hline
  \rowcolor[HTML]{EFEFEF} 
  \textsc{$\delta$-RGVAE permutation}\\ \hline
  \texttt{[[The Gift] [/user/tsegaran/random/taxonomy] [Alyson Hannigan]]}\\
  \texttt{[[Philip K. Dick Award] [/people/deceased] [CSI: Crime Scene Investigation]]}\\
  \texttt{[[The Crying Game] [/user/tsegaran/random/taxonomy] [Stephen Tobolowsky]]}\\
  \texttt{[[Anne Hathaway] [/user/tsegaran/random/taxonomy] [Billie Joe Armstrong]]}\\
  \texttt{[[Mehcad Brooks] [/user/tsegaran/random/taxonomy] [Melissa Leo]]}\\
  \texttt{[[Soap] [/user/tsegaran/random/taxonomy] [United States of America]]}\\
  \texttt{[[2002 Winter Olympics] [/user/tsegaran/random/taxonomy] [Swept Away]]}\\
  \texttt{[[Oliver Platt] [/user/tsegaran/random/taxonomy] [The Good] [the Bad] [the Weird]]}\\
  \texttt{[[Danny DeVito] [/user/tsegaran/random/taxonomy] [Percussion]]}\\
  \texttt{[[Mr. \& Mrs. Smith] [/user/tsegaran/random/taxonomy] [Marie Antoinette]]}\\  \hline
  \end{tabular}
\caption{RGVAE latent space interpolation with $\delta = 0.6$ and graph matching.}
\label{tab5:ipbtw2Delta}
\end{table}

% show parameters if different
Finally, we plot the parameter values, both weight and bias to visualize the collapsing decoder. The values are logscaled and plotter per layer number of encoder and decoder as described in table \ref{tab4:archcompare}. The parameters of the decoder behave equally for both $\delta$ values. All values of the 3rd layer conglomerate around zero revealing a bottleneck of the information stream in the decoder. This indicates that the problem of collapsing decoder was not solved by the addition of the new parameter. The parameter distribution for the RGVAE with standard loss are appended in the annex \ref{annexC:deltaParamsP0} and \ref{annexC:normParamsP0}. The impact of the standard loss function is a more narrow distributed decoder. Since the GCN encoder has different layers, a direct parameter comparison to the MLP encoder is not possible, yet we observe the same parameter distribution for the decoder. Note in figure \ref{fig5:deltaParams} the different scales of the value axis, which are $10^2$ larger for $\delta = 0.6$ than for $\delta = 0$.

\begin{figure}[H]
  \centering
  \begin{subfigure}{\textwidth}
    \includegraphics[width=\textwidth]{/home/wolf/Thesis/Code/Thesis/data/ip/scatter_GVAE_fb15k_p1_delta_Encoder_weight.png}
    \caption{Weight}
    \label{fig5:deltaParamsW}
  \end{subfigure}
  \begin{subfigure}{\textwidth}
    \includegraphics[width=\textwidth]{/home/wolf/Thesis/Code/Thesis/data/ip/scatter_GVAE_fb15k_p1_delta_Encoder_bias.png}
    \caption{Bias}
    \label{fig5:deltaParamsB}
  \end{subfigure}
\caption{Parameter values per layer of the RGVAE encoder and decoder with $\delta=0.6$.}
\label{fig5:deltaParams}
\end{figure}


\begin{figure}[H]
  \centering
  \begin{subfigure}{\textwidth}
    \includegraphics[width=\textwidth]{/home/wolf/Thesis/Code/Thesis/data/ip/scatter_GVAE_fb15k_p1_Encoder_weight.png}
    \caption{Weight}
    \label{fig5:normParamsW}
  \end{subfigure}
  \begin{subfigure}{\textwidth}
    \includegraphics[width=\textwidth]{/home/wolf/Thesis/Code/Thesis/data/ip/scatter_GVAE_fb15k_p1_Encoder_bias.png}
    \caption{Bias}
    \label{fig5:normParamsB}
  \end{subfigure}
\caption{Parameter values per layer of the RGVAE encoder and decoder with $\delta=0$.}
\label{fig5:normParams}
\end{figure}


% \section{Results}
% \input{sections/section6}

\section{Discussion \& Future Work}
We will discuss certain aspects of our results and give advice on further research.

\subsection{Discuss Aspect 1}

Well well well

\subsection{Future Work}

There are many avenues to follow for future work.

Basing on the believe that the experiments were successful, we recommend:

\begin{itemize}
    \item Improve the model (deeper)
    \item link the latent space to word signals e,g text
    \item prior not normal, e.g NF
    \item try a GAN
\end{itemize}

\textbf{Good luck amigos!}

\label{sec:discus}

% \section*{Ideas}
% 

The bigger picture of this thesis is to efficiently generate a representation of the information hold in plain text. This representation has the form of a knowledge graph and consists of subject-relation-object triples.
\\
\textbf{Challenges:}\\
\begin{itemize}
    \item What knowledge are we looking for?
    I would like the model to focus on the most important information. This could for example be topic specific.
    \item Another option would be to extract based on a query or point of interest.
    Especially if we build the KG incrementally we can use this input as starting point and recursively build from there. 
\end{itemize}


\subsection{Use Normalizing Flows}\label{idea:NF}

\begin{itemize}
    \item Is it possible to generate a graph from text using Normalizing flow (NF)?
    \item Can we train such a NF unsupervised, using the output distributions as loss function?
    \item NFs can build KG at once or incrementally.
No one is using it, could have a reason.
\end{itemize}

    
\subsection{Recurrent VAE}\label{idea:VAE}

\begin{itemize}
    \item Recursively generate graph using a Variational Auto Encoder.
    \item Use query as start node.
    \item expand graph on that node.
Thiviyan and others work on VAEs as well.
Seems an easier approach.
Can analyze the latent space.

\end{itemize}

The encoder decoder strategy has been applied successfully in many cases.
Bellis thesis about modeling a graph from images can be applied to text as well by changing the input to a text vector \cite{belli_image-conditioned_2019}. Note that this is a supervised method and would require a dataset of text labeld with resulting KGs. To the best of my knowledge this does not exist yet.
\\
Same holds for the contrastive world model by Kipf. If we input text vectors instead of an image the model could recognize different objects in the text as it does with pixels \cite{kipf_contrastive_2020}. Here the model is trained unsupervised using a loss over the energy function of the graph embedding space TransE \cite{bordes_translating_2013}. An open question is if this would work for our approach.
\\
A bit more abstract is the idea of the feedback recurrent VAE \cite{yang_feedback_2020} where sound signals are encoded and decoded. This could also be adopted to text for instance with one sentence at a time, or a fixed number of tokens. Here the text vector would be induced as the latent input to the decoder. This would mean finding an translation of the models latent space to the work embedding. The dataset would consist of positive and negative examples of resulting graphs over timesteps.
\\
While text can be easily vectorized by word embeddings like word2vec, the graph representation seems more tricky. A reasonable approach following Bellis example would be to output a coordinates vector for the nodes and an adjantency matrix for their relationships. Here one node at a time is outputted and the relation is conditioned on the number of previous nodes.
Alternatively we could make use of the graph embeddings RASCAL or TransE.
Lastly the question remains how to model the graph when the nodes need to be predefined. A subgraph of DBpedia could be a good starting point.

% \section*{Open Issues}
% A section where note will be made on open issus. These aissues are ment to be discussed with Peter or Thivyian and ultimatly solved. No open issues should remain at the end of November.


\subsection{Graph matching}

NetworkX offers graph matching algorithms as do other repos. The drawback is that they are programmed for smaller graph such as molecule graphs with undirected edges and no edge atribues, or these attributes are not taken into account and only the structure is compared.
My implementation of the max-pooling graph matching algorithm from \cite{simonovsky_graphvae_2018} return a $n\times n\times k\times k$ matrix with integer values ${0,2}$. Most probably something went wrong here since this makes little sense regarding interpretation.
When graphs are compared with NetworkX using the greedy-matching or minimum distance, the similarity matrix is a $2\times 2$ matrix for comparing two graphs.
My question is which sort of graph matching makes more sense for our usecase.
Further I struggle to interprete the similarity matrix - is that my job tho? Maybe its just meant to be used in the loss function as described! 

\printbibliography
\end{document}

% \bibliographystyle{unsrt} % We choose the "plain" reference style
% \bibliography{fullrefs} % Entries are in the "refs.bib" file

