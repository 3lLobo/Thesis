In this section we will go over related work and relevant background information for our model and experiments. The depth of the explanation is adopted to the expected prior knowledge of the reader. The reader is supposed to know the basics of machine learning and deep learning, including probability theory and basic knowledge on neural networks and their different architectures. Basic principles such as forward pass, backpropagation and convolutions are expected to be understood. Further the use and functionality of deep learning modules such as the model, the optimizer and the terms target and prediction should be known. This also includes being familiar with the training and testing pipeline of a model in deep learning.

Should all these boxes be checked, then we can expect to get a deeper understanding of the magic behind the VAE and its differences to a normal autoencoder. After that we will present how convolutional layers can be used on graphs. Of course where there is a layer there is a model, thus we are presenting the graph convolutional network (GCN). Closing the circle we show how we can adopt the VAE to graph convolutions. Wrapping things up, we present the state of the art algorithms for graph matching, which will be util to allow permutation invariance when matching prediction and target \cite{paulheim_knowledge_2016}.  


\subsection{Knowledge Graph}

% Knowledge Graphs are great! The best in the world.
% Knowledge Graphs which we will be using
% We will focus on the generation of KGs.
% Representation of KG as adjacency, edge feature and node feature matrix
Knowledge graph has become a popular key phrase. Yet, the term is so broad, that it can have various definitions. In this thesis we are going to focus on KGs in the context of relational machine learning.

% What is a KG
A KG is a database and as all other databases it is used to store data. The main difference to tabular databases is that KGs store data in a relational fashion. A standard KG structure, introduced by the semantic-web community, is the Resource
Description Framework (RDF). It is a so called schema-based approach, meaning that every entity has a unique identifier and all possible relations are stored in a vocabulary. The opposite schema-free approach is used in OpenIE models for information extraction. Here any type of triple can be extracted, eg.  \text{(Michelangelo,  painted,    Sixtine Chapel)}. Where as a triple in RDF format from the Freebase KG has the form
\begin{equation}
    (/m/02mjmr, /people/person/born-in, /m/03gh4)
\end{equation}
%  
\begin*{equation}
    (s, r,  o)
\end{equation}
 
with:
\begin{itemize}
    \item Subject $s:   /m/02mjmr$ Barack Obama
    \item Relation/Predicate $r:    /people/person/born-in$
    \item Object $o:     /m/03gh4$ Hawaii
\end{itemize}


For all triples $s$ and $o$ are part of a set of so called entities, while $r$ is part of a set of relations. This is enough to define a basic KG.

% Hierarchy, entities, classes
Schema-based KG can include type hierarchies and type constrains. Classes group entities of the same type together, based on common criteria, e.g all names of people can be grouped in the class 'person'.
Hierarchies define the inheriting structure of classes and subclasses. Picking up our previous example, 'child' would be a subclass of 'person' and inherit its properties. At the same time the class of an entity can be key to a relation with type constrain, since some relations can only be used in conjunction with entities fulfilling the constraining type criteria.

% Ontology - semantics
These schema based rules of a KG are defined in its ontology. Here properties of classes, subclasses and constrains for relations and many more are defined. Again we we have to differentiate here between KGs with open-world or closed-world assumption. In a closed-world assumption all constrains must be sufficiently satisfied before a triple is accepted as valid. This leads to a huge ontology and makes it difficult to expand the KG. On the other hand open-world KGs such as Freebase, accept every triple as valid, as long as it does not violate a constrain. This leads inevitably to inconsistencies within the KG, yet it is the preferred approach for large KGs. In context of this thesis we refer to the ontology as semantics of a KG, we research if our model can capture the implied closed-world semantics of an open-world KG \cite{nickel_review_2016}.

% Sparse and dense representation
Lastly, we point out one major difference between KGs, namely their representation. RDF KGs are represented as set of triples, consisting of a unique combination of numeric indices. Each index linking to the corresponding entry in the entity and relation vocabulary. This is called dense representation and benefits from fast computation due to an optimized use of memory.

In contrary the dense representation of a triple is the sparse representation. Here a binary square matrix also called the adjacency matrix, indicated a link between two entities. To identify the node, each node in the adjacency matrix has a one-hot encoded entity-vocabulary vector. All one-hot encoded vectors are concatenated to a node attribute matrix.
In simple cases, like citation networks this is a sufficient representation. In the case of Freebase, we need an additional edge-attribute matrix, which indicates the relation  of each link. The main benefits of this method are the representation of subsets of triples, also subgraphs, with more than one relation and the computational possibility to perform graph convolutions. 

% Usecases of KG

% Knowledge graphs have very different formats. The datasets we will be sorting with are in rdf format.
% This format can include an defined onthology or not.
% This means the KG consists of triples subject, relation, object.
% when indexing these triples. we get a dense representation of the KG.
% About sparse KGs

\subsection{Graph VAE – one shot method}

\subsubsection{VAE}
% The VAE as first presented by \cite{kingma_auto-encoding_2014} is an unsupervised generative model consisting of an encoder and a decoder. The architecture of the VAE differs from a common autoencoder by having a stochastic module between encoder and decoder. Instead of directly using the output of the encoder, a distribution of the latent space is predicted from which we sample the input to the decoder. The reparameterization trick allows the model to be differentiable. By placing the sampling module outside the model we get a deterministic model which can be backpropagated.



The VAE as first presented by \cite{kingma_auto-encoding_2014} is an unsupervised generative model in form of an autoencoder, consisting of an encoder and a decoder. We will The architecture of the VAE differs from a common autoencoder by having a stochastic module between encoder and decoder. The encoder can be represented as recognition model with the probability $p_{\boldsymbol{\Theta}}(\mathbf{z} \mid x)$ with $x$ being the varbiable we want to inference and $z$ being the latent representation given an observed value of $x$. The encoder parameters are represented by $\Theta$. Sinilarly, We denote the decoder as $p_{\boldsymbol{\Theta}}(\mathbf{x} \mid z)$, which given a latent representation $z$ produces a probability distribution for the possible values, corresponding to the input of $x$. This will be the base architecture of all our models in this thesis.

The main contribution of the VAE is the so called reparameterization trick. By sampling from the latent prior distribution, we get stochastic module inside our model, which can not be backpropagates through and makes machine learning not possible. By placing the stochastic module outside the model FIGURE!!!, we can again backpropagate. We use the predicted latent space as mean and variance for a Gaussian normal distribution, from which we then sample $\epsilon$, which acts as external parameter and does not need to be updated.

This makes the true posterior $p_{\boldsymbol{\theta}}(\mathbf{z} \mid \mathbf{x})$ intractable. Thus, we assume that the prior to the decoder to be Gaussian with an approximately diagonal covariance, which gives us the approximated posterior.

\begin{equation}
    \log q_{\phi}\left(\mathbf{z} \mid \mathbf{x}^{(i)}\right)=\log \mathcal{N}\left(\mathbf{z} ; \boldsymbol{\mu}^{(i)}, \boldsymbol{\sigma}^{2(i)} \mathbf{I}\right)
\end{equation}
    
This gives us computational freedom. meaning we can compute the posterior probability. Using the Monte Carlo estimation of $q_{\phi}(\mathbf{z} \mid \mathbf{x})$ we get the so called estimated lower bound (ELBO):
\begin{equation}
    y_{k}(\mathbf{x}, \mathbf{w})=\sigma\left(\sum_{j=0}^{M} w_{k j}^{(2)} h\left(\sum_{i=0}^{D} w_{j i}^{(1)} x_{i}\right)\right)
\end{equation}
We denote the first term the regularization term, as it forces the model into using a Gaussian normal prior. The second term represents the reconstruction loss, matching the prediction with the target.

While we use a discrete input space, the output space is a continuous probability. To generate final result, the prediction is used as binomial probability distribution, from which we then sample. Once training  of a VAE is completed, the Decoder can be used on its own to generate new samples by using latent input signals \cite{kingma_auto-encoding_2014}.


\subsubsection{MLP}
% History introduction
The Multi-Layer Perceptron (MLP) was the beginning of machine learning models.
% Invented by who when
Its properties as universal approximator has been discovered and widely studied since 1989. The innovation it brought to existing models was the hidden layer between the input and the output.

% Functionality
% In its basic structure it takes a one dimensional input, fully-connected hidden layer, activation function and finally output layer with normalized predictions.
The mathematical definition of the MLP is rather simple. It takes linear input vector of the form $x_1,...,x_D$ which is multiplied by the weight matrix $\mathbf{w^{(1)}}$ and then transformed using a non-linear activation function $h(\dot$. Due to its simple derivative, mostly the rectified linear unit (ReLU) function is used. This results in the hidden layer, consisting of hidden units. The hidden units get multiplied with the second weight matrix, denoted $\mathbf{w^{(2)}}$ and finally transformed by a sigmoid function $\sigma(\dot)$, which produces the output.Grouping all weight and bias parameter together we get the following equation for the MLP:

\begin{equation}
    y_{k}(\mathbf{x}, \mathbf{w})=\sigma\left(\sum_{j=1}^{M} w_{k j}^{(2)} h\left(\sum_{i=1}^{D} w_{j i}^{(1)} x_{i}+w_{j 0}^{(1)}\right)+w_{k 0}^{(2)}\right)
\end{equation}

for $j=1, \ldots, M$ and $k=1, \ldots, K$, with $M$ being the total number of hidden units and $K$ of the output.


% Application
Since the sigmoid function gives us a probability output, the main function of the MLP is as a classifier. Instead of the initial sigmoid function, it was found to also produce good results for multi label classification transforming the output with a softmax function instead. Images or higher dimensional tensors can be processed by flattening them to a one dimensional tensor. This makes the MLP a flexible and easy to implement model \cite{bishop2006pattern}.

\\
\subsubsection{Graph convolutions}\\

% TODO pretty write this
%  Intro on convolutions
Convolutional neural nets (CNN) have the advantage to be invariant to permutations od the input. Convolutional layers exploit the property of datapoints which are close to each other and thus, have a higher correlation. 

CNNs have shown great results in the field of images classification and object detection. Neighboring pixel contain information about each other which can let the model detect local features which can then be merged to high-level features \cite{bishop2006pattern}.

Similar conditions hold for graphs. Neighboring nodes contain information about each other and can used to infer local features. 

% How do Graph convs work???
Let us shortly go over the definition and math behind graph convolutions. Different approaches have been published on this topic, here we will present the graph convolution network (GCN) of \cite{kipf_semi-supervised_2017}.
We consider $f(X,A)$ a GCN with an undirected graph input $\mathcal{G}=(\mathcal{V}, \mathcal{E})$, where $v_{i} \in \mathcal{V}$ is a set of $N$ nodes and  $\left(v_{i}, v_{i}\right) \in \mathcal{E}$ the set of edges. The input is a sparse graph representation, with $X$ being a node feature matrix and $A \in \mathbb{R}^{N \times N}$ being the adjacency matrix, defining the position of edges between nodes. In the initial case, where self-connections are not considered, the adjacency's diagonal has to be one resulting in $\vec{A}=A+I_{N}$. The graph forward pass through the convolutional layer $l$ is then defined as

% equation
\begin{equation}
    H^{(l+1)}=\sigma\left(\tilde{D}^{-\frac{1}{2}} \tilde{A} \tilde{D}^{-\frac{1}{2}} H^{(l)} W^{(l)}\right).
\end{equation}

Here $\tilde{D}_{i i}=\sum_{j} \tilde{A}_{i j}$ acts as normalizing constant. $W^{(l)}$ is the layer-specific weight matrix and contains the learnable parameters. $H$ returns then the hidden representation of the input graph.

The GCN was first introduced as node classifier, meaning it returns a probability distribution over all classes for each node in the input graph $\mathcal{V}$. Assuming that we preprocess $\vec{A}$ as $\hat{A}=\tilde{D}^{-\frac{1}{2}} \vec{A} \tilde{D}^{-\frac{1}{2}}$, the equation for a two-layer GCN for $z$ classes is

\begin{equation}
    Z=f(X, A)=\operatorname{softmax}\left(\hat{A} \operatorname{ReLU}\left(\hat{A} X W^{(0)}\right) W^{(1)}\right).
\end{equation}

\\
\subsubsection{RGCN}

GCN which takes further input of edge attribute matrix.

Either present\\
Dynamic Edge-Conditioned Filters in Convolutional Neural Networks on Graphs\\
or nixx
% Realational Graph Convolution Net (RGCN) was presented in \cite{kipf_semi-supervised_2017} for edge prediction. This model takes into account features of nodes. Both the adjacency and the feature matrix are matrix-multiplied with the weight matrix and then with them-selves. The resulting vector is a classification of the nodes.
\\
\subsubsection{Graph VAE}
% Encoder options: MLP RGCN
% Decoder MLP
% One shot: creating adjacency and feature matrix at once.

% the model we are going to use for challenging KG datasets
Now we have all the building bocks for a Graph VAE. The encode can either be a MLP, a GCNN or an RGCN. The same holds for the decoder with the addition that model architecture needs to be inverted. An verison of a Graph VAE presented in \cite{simonovsky_graphvae_2018}. This model combines both the previous methods. The input graph undergoes relational graph convolutions before it is flattened and projected into latent space. After applying the reparametrization trick, a simple MLP decoder is used to regenerate the graph. In addition the model concatenates the input with a target vector $y$, which represents ???. The same vector is concatenated with the latent tensor. ***Elborate why they do that***.

% Reverse MLP

% Loss function

% What information can you capture when sparse compared to dense?


Graphs can be generated recursively or in an one-shot approach. This paper uses the second approach and generates the full graph in one go. ***Cite?***
% This model will be the starting point for our research.

\subsubsection{One Shot vs. Recursive}
% One shot: MNIST vs recursive on graphs: Belli
The concept of the VAE has been used to generate data for various usecases. When using the VAE as generator, by sampling from the approximated posterior distribution $q_{\phi}\left(\mathbf{z}$, we can reconstruct the data in a singular run or recursive manner.

The one shot method is the used in the popular example of the VAE generator on the MNIST daataset [cite], as well as on [the faces dataset]. each sample is independent from each other.

Recursive methods take part of the generated datapoint as input for the next datapoint, thus continuously generating the sample. This has been appied to [voice] and to reproduce videogame environments \cite{ha_world_2018}. In \cite{belli_image-conditioned_2019} variation of the GraphVAE has been used to recursively construct a vector roadmap, which has been presented in [link].

For this thesis, we will use the one-shot method, predicting each datapoint independent from each other. The predictions will sparse graph representation with $n$ nodes. A single triple being $n=2$ and a subgraph representation $2<n<100$. [TODO]


\subsection{Graph Matching}
Intro to graph matching on sparse graphs.

\subsubsection{Permutation Invariance}

% Permutation Invariance
% The position or rotation of a graph can vary. 
% Use graph matching to detect similarities between graphs

Permutation invariance refers to the invariance of a permutation of an object. An visual example is the the image generation of numbers. If the loss function of the model would not be permutation invariant, the generated image could show a perfect replica of the input number but due to positional permutation the loss function would penalize the model. 
OR: An example is in object detection in images. An object can have geometrical permutations such as translation, scale or rotation, none the less the model should be able to detect and classify it. In that case, the model is not limited by permutations and  is there fore permutation invariant.
In our case the object is a graph and the nodes can take different positions in the adjacency matrix. To detect similarities between graphs we apply graph matching.

\subsubsection{Max-Pool Graph matching algorithm}
These are three of the state of the art graph matching algorithms.

% \begin{itemize}
%     \item Wasserstein
%     \item Maxpooling
%     \item one more
% \end{itemize}

Can be used for graphs with different number of nodes.
% present different ones or only max-pooling?
There are various graph matching algorithms. The one we will implement is the max-pooling (Finding Matches in a Haystack: A Max-Pooling Strategy for Graph Matching in the Presence of Outliers). 

% Max-pooling algorithm comes here !!!
The max-pooling graph matching algorithm returns a symmetric affinity matrix for all nodes

The resulting similarity matrix gives us $X*$ which is continuous and therefore useless. To transform is to a discrete $X$ we use the hungarian algorithm (GPU-accelerated Hungarian algorithms for the Linear
Assignment Problem)

\textbf{Hungarian algorithm}\\
The hungarian algorithm is used to find the shortest path within a matrix. This could be the most efficient work-distribution in a cost matrix. Or \dots \\
It consists of four steps of which the last two are repeated until convergence. This algorithm is not scalable. The Munks algorithm (reference) tackles this problem by?? and is scalable.

second option: \leavevmode
Compare only graph structure. 
NX algorithms: Greedy, Shortest path \dots
\\
\subsubsection{Graph Matching Loss}
The loss is discribed in \cite{simonovsky_graphvae_2018} as.
% \begin{align*}
%     -\log p(G \mid \mathbf{z}) &=-\lambda_{A} \log p\left(A^{\prime} \mid \mathbf{z}\right)-\lambda_{F} \log p(F \mid \mathbf{z})-\\
%     &-\lambda_{E} \log p(E \mid \mathbf{z})
%     &\log p\left(A^{\prime} \mid \mathbf{z}\right)=\\
%     &=1 / k \sum_{a} A_{a, a}^{\prime} \log \widetilde{A}_{a, a}+\left(1-A_{a, a}^{\prime}\right) \log \left(1-\widetilde{A}_{a, a}\right)+\\
%     &+1 / k(k-1) \sum_{a \neq b} A_{a, b}^{\prime} \log \widetilde{A}_{a, b}+\left(1-A_{a, b}^{\prime}\right) \log \left(1-\widetilde{A}_{a, b}\right)\\
%     &\log p(F \mid \mathbf{z})=1 / n \sum_{i} \log F_{i, \cdot}^{T} \tilde{F}_{i,}^{\prime}\\
%     &\log p(E \mid \mathbf{z})=1 /\left(\|A\|_{1}-n\right) \sum_{i \neq j} \log E_{i, j}^{T}, \widetilde{E}_{i, j, \cdot}^{\prime}
% \end{align*}

In contrast to his implementation we assume, that a node or edge can have none, one or multiple attributes. Therefore our attributes are also not sigmoided and do not sum up to one. This leads o the modification of term logF and logE where we do not matrix multiply over the attribute vector but take the BCE as over the rest of the points.
KG can have multiple or no attributes vs molecular graphs can be one hot encoded.

Okay, further we need to treat the $log_pE$ and $log_pF$ just like $log_pA$ and subtract the inverse. Otherwise the model learn to predict very high values only. 

A note to the node features, these stay softmaxed and one-hot encoded since we will use them as node labels.


\subsection{Ranger Optimizer}

Finalizing this chapter, we introduce the novel optimizer Ranger. A optimizer, which placed itself on the top of 12 FastAI leaderboards.Ranger combines Rectified Adam (RAdam), lookahead and optionally gradient centralization. let us shortly look into the different components.

%  REadam
RAdam is based on the very popular adam optimizer. It improves the learning by dynamically rectifying Adam's adaptive momentum. It does this by reducing the variance of the momentum, which is especially large at the beginning of the training. Thus, leading to a more stable and accelerated start \cite{liu_variance_2020}.

The Lookahead optimizer proposes an approach where, a second optimizer 'looks ahead' on a set of parallel trained weights. while the computation and memory cost are negligible, it learning improves and the variance of the main optimizer is reduced \cite{zhang_lookahead_2019}.

The last and most novel optimization technique, Gradient Centralization, acts directly on the gradient by normalizing it to a zero mean. Especially on Deep Neural Networks, this helps regularizing the gradient and boosts learning. This method can be added to existing optimizers and can be seen as constrain of the loss function \cite{yong_gradient_2020}.

Concluding we can say that Ranger is a state of the art deep learning optimizer with accelerating and stabilizing properties, incorporating three different optimization methods, which synergize with each other. Considering that generative models are especially unstable during training, we see Ranger as a good fit for this research.









LookAhead was inspired by recent advances in the understanding of loss surfaces of deep neural networks, and provides a breakthrough in robust and stable exploration during the entirety of training.