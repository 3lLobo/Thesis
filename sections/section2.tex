This section presents previous work which inspired and layed the fundamentals for this thesis.

\subsection{Graph VAE}

Present different papers with graph VAEs

\begin{itemize}
    \item Belli recurrent VAE
    \item GraphVAE paper
    \item some more
\end{itemize}

The model architecture presented in the GraphVAE paper is the starting point of our model.

\subsection{RGCN}

Present the relational graph convolution model paper by Kipf and maybe others

\subsection{Embedding models}

Present RASCAL and one or two more.

\Graph Embeddings\\
TransE represents entities in in low-dimensional embedding. The relationships between entities are represented by the vector between two entities \cite{bordes_translating_2013}.
(How are different relation between the same entities represented?)

OntoUSP\\
This method learns a hierarchical structure to better represent the relations between entities in embedding space.
